\documentclass[a4paper,12pt]{article}
\usepackage{geometry}
\geometry{left=2.5cm,right=2.5cm,top=2.5cm,bottom=2.5cm}
\renewcommand{\textfraction}{0.15}
\renewcommand{\topfraction}{0.85}
\renewcommand{\bottomfraction}{0.65}
\renewcommand{\floatpagefraction}{0.60}
\usepackage{amsmath}
\usepackage{amsfonts}
\usepackage{mathrsfs}
\usepackage{amsthm}
\usepackage{extarrows}
\usepackage{bm}
% \newcommand{\bm}{\symbfit}    % `bm` confilicts with `unicode-math`. In that case use \symbfit for bold math symbols
\usepackage{graphicx}
\usepackage[section]{placeins}
\usepackage{flafter}
\usepackage{array}
\usepackage{caption}
\usepackage{subcaption}
\usepackage{color}
\usepackage{multirow}
\usepackage{natbib}
% \usepackage{enumerate}
\usepackage{enumitem}    % more flexible than `enumerate` package, the reference will carry the whole label appearance, not just the counter, unlike the `enumerate` package.
\usepackage{upgreek}    % 'upgreek' letters

% \pdfstringdefDisableCommands{\let\bm=\relax}

\newtheorem{thm}{Theorem}
\newtheorem{cor}{Corollary}
\newtheorem{assum}{Assumption}
\newtheorem{rem}{Remark}
\newtheorem{lem}{Lemma}

\setcounter{topnumber}{5}    % Maximum number of floats that can appear at the top of a text page; default 2. 
\setcounter{bottomnumber}{5}   % Maximum number of floats that can appear at the bottom of a text page; default 1. 
\setcounter{totalnumber}{10}    % Maximum number of floats that can appear on a text page; default 3. 

\DeclareMathOperator*{\argmaxdown}{arg\,max}
\DeclareMathOperator*{\argmindown}{arg\,min}
\DeclareMathOperator{\argmax}{arg\,max}
\DeclareMathOperator{\argmin}{arg\,min}

% cross-reference to other files
% the externaldocument should be compiled 
% (and at least twice if you're using xr-hyper)
% \usepackage{xr-hyper}
% \usepackage{xr}

% --- external document (ordinary setting) ---
% \externaldocument{external_tex_file}
% --- end of ordinary setting ---

% --- external document (overleaf setting) ---
% externaldocument settings for Overleaf
% \makeatletter
% \newcommand*{\addFileDependency}[1]{% argument=file name and extension
% \typeout{(#1)}
% \@addtofilelist{#1}
% \IfFileExists{#1}{}{\typeout{No file #1.}}
% }
%   \makeatother
%   \newcommand*{\myexternaldocument}[1]{%
%   \externaldocument{#1}%
%   \addFileDependency{#1.tex}%
%   \addFileDependency{#1.aux}%
% }
%   \myexternaldocument{external_tex_file}
%   --- end of overleaf setting ---

%   mathtools can be used to define labeling format for equations
%   one can use \eqref for a reference to a labeled equation.
\usepackage{mathtools}
\newtagform{supp}{(S-}{)}    % define a equation labeling format for suppliment
\usetagform{supp}            % use the supp format
\usetagform{default}         % use the default format

\usepackage{algorithmic}
\usepackage{algorithm}
\renewcommand{\algorithmicrequire}{\textbf{Input:}}
\renewcommand{\algorithmicensure}{\textbf{Output:}}

% package for hyperlinks
% It's error-prone because hyper link is quite difficult
% due to the fact the typesetting environment is complex.
% So you can disable this package and finish the document.
% Then sort out the hyperlink thing.
\usepackage[colorlinks,linkcolor=red,anchorcolor=blue,citecolor=green,CJKbookmarks=True]{hyperref}

% package for displaying highlighted codes
\usepackage{minted}

% package for input codec and output rendering font
\usepackage[utf8]{inputenc}    % it is always good practice to use utf8
% you can also try latin1, latin2, cp1252 and cp1250
\usepackage[T1]{fontenc}    % the default is T0, which only contains 128 characters
% you can try T1, T2A, T2B
% you can refer to https://www.overleaf.com/learn/latex/international_language_support#Font_encoding
% for more details.


\title{Gamma and Beta Distribution}
\author{Chao Cheng}
\date{\today}



\begin{document}
\maketitle

\tableofcontents{}

\section{Introduction}
\label{sec:introduction}

The Gamma function is defined as
\begin{equation}
  \label{eq:gamma_def}
  \Gamma\left(\alpha\right)
  = \int_0^\infty t^{\alpha - 1}e^{-t}\mathrm{d}t
  .
\end{equation}
And it's easy to verify that
\[
  \Gamma\left(\alpha + 1\right) = \alpha\Gamma\left(\alpha\right)
  ,\quad
  \Gamma\left(n\right) = \left(n - 1\right)!
  ,\quad
  \Gamma\left(0.5\right) = \sqrt{\pi}
  .
\]

Let $B\left(\alpha, \beta\right)$ denote the Beta function
\begin{equation}
  \label{eq:beta_def}
  B\left(\alpha, \beta\right) =
  \int_0^1x^{\alpha - 1}\left(1 - x\right)^{\beta - 1}\mathrm{d}x
  .
\end{equation}
Here we point out that $B\left(\alpha, \beta\right)$ is related to Gamma function via
\[
  B\left(\alpha, \beta\right) =
  \frac{\Gamma\left(\alpha\right)\Gamma\left(\beta\right)}{\Gamma\left(\alpha + \beta\right)}
  .
\]

\section{Gamma distribution}
\label{sec:gamma-distribution}

The pdf for a Gamma distribution is
\begin{equation}
  \label{eq:gamma_pdf}
  f\left(x\middle|\alpha, \beta\right)
  = \frac{1}{\Gamma\left(\alpha\right)\beta^\alpha}
  x^{\alpha - 1}
  \mathrm{exp}\left(-x / \beta\right)
  ,\quad
  0 \leq x < \infty
  ,\quad
  \alpha, \beta > 0
  .
\end{equation}

\section{Beta distribution}
\label{sec:beta-distribution}

The pdf for a Beta distribution is
\begin{equation}
  \label{eq:beta_pdf}
  f\left(x\middle|\alpha, \beta\right) =
  \frac{1}{B\left(\alpha, \beta\right)}
  x^{\alpha - 1}\left(1 - x\right)^{\beta - 1}
  ,\quad
  0 \leq x \leq 1
  ,\quad
  \alpha, \beta > 0
  .
\end{equation}

\section{Relationship with other distribution}
\label{sec:relat-other-distr}

\begin{itemize}
\item Gamma distribution with $\alpha = 1$ is just exponential distribution.
  \[
    gamma\left(1, \beta\right) = \mathrm{exp}\left(\beta\right)
    .
  \]
  
\item Gamma distribution with $\beta = 2$ is just $\chi^2$ distribution with degree of freedom $p = 2\alpha$.
  \[
    gamma\left(\alpha, 2\right) = \chi^2\left(2\alpha\right)
    .
  \]
  
\item If $X$ follows $gamma\left(\alpha, \beta\right)$ where $\alpha$ is an integer. Then for any $x$
  \[
    P\left(X \leq x\right) = P\left(Y \geq \alpha\right)
    .
  \]
  where $Y$ follows $Poisson\left(x / \beta\right)$.
  
\item If
  \[
    X \sim gamma\left(\alpha_1, \beta\right)
    ,\quad
    Y \sim gamma\left(\alpha_2, \beta\right)
    ,\quad
    X \perp Y
    .
  \]
  Then
  \[
    X + Y \sim gamma\left(\alpha_1 + \alpha_2, \beta\right)
    ,\quad
    \frac{X}{X + Y} \sim beta\left(\alpha_1, \alpha_2\right)
    ,\quad
    \left(X + Y\right) \perp \frac{X}{X + Y}
    .
  \]
\item
  \[
    beta\left(1, 1\right) = Unif\left(0, 1\right)
    .
  \]
  
\item $beta\left(\frac{1}{2}, \frac{1}{2}\right)$ is the \textbf{non-informative} prior for binomial rate test.
\end{itemize}

\bibliographystyle{plainnat}
\bibliography{../ref}


\end{document}



%%% Local Variables:
%%% mode: latex
%%% TeX-master: t
%%% End:
