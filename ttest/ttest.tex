\documentclass[a4paper,12pt]{article}
\usepackage{geometry}
\geometry{left=2.5cm,right=2.5cm,top=2.5cm,bottom=2.5cm}
\renewcommand{\textfraction}{0.15}
\renewcommand{\topfraction}{0.85}
\renewcommand{\bottomfraction}{0.65}
\renewcommand{\floatpagefraction}{0.60}
\usepackage{amsmath}
\usepackage{amsfonts}
\usepackage{mathrsfs}
\usepackage{amsthm}
\usepackage{extarrows}
\usepackage{bm}
% \newcommand{\bm}{\symbfit}    % `bm` confilicts with `unicode-math`. In that case use \symbfit for bold math symbols
\usepackage{graphicx}
\usepackage[section]{placeins}
\usepackage{flafter}
\usepackage{array}
\usepackage{caption}
\usepackage{subcaption}
\usepackage{color}
\usepackage{multirow}
\usepackage{natbib}
% \usepackage{enumerate}
\usepackage{enumitem}    % more flexible than `enumerate` package, the reference will carry the whole label appearance, not just the counter, unlike the `enumerate` package.
\usepackage{upgreek}    % 'upgreek' letters

% \pdfstringdefDisableCommands{\let\bm=\relax}

\newtheorem{thm}{Theorem}
\newtheorem{cor}{Corollary}
\newtheorem{assum}{Assumption}
\newtheorem{rem}{Remark}
\newtheorem{lem}{Lemma}

\setcounter{topnumber}{5}    % Maximum number of floats that can appear at the top of a text page; default 2. 
\setcounter{bottomnumber}{5}   % Maximum number of floats that can appear at the bottom of a text page; default 1. 
\setcounter{totalnumber}{10}    % Maximum number of floats that can appear on a text page; default 3. 

\DeclareMathOperator*{\argmaxdown}{arg\,max}
\DeclareMathOperator*{\argmindown}{arg\,min}
\DeclareMathOperator{\argmax}{arg\,max}
\DeclareMathOperator{\argmin}{arg\,min}

% cross-reference to other files
% the externaldocument should be compiled 
%   (and at least twice if you're using xr-hyper)
% \usepackage{xr-hyper}
% \usepackage{xr}

% --- external document (ordinary setting) ---
% \externaldocument{external_tex_file}
% --- end of ordinary setting ---

% --- external document (overleaf setting) ---
% externaldocument settings for Overleaf
% \makeatletter
% \newcommand*{\addFileDependency}[1]{% argument=file name and extension
%   \typeout{(#1)}
%   \@addtofilelist{#1}
%   \IfFileExists{#1}{}{\typeout{No file #1.}}
% }
% \makeatother
% \newcommand*{\myexternaldocument}[1]{%
%     \externaldocument{#1}%
%     \addFileDependency{#1.tex}%
%     \addFileDependency{#1.aux}%
% }
% \myexternaldocument{external_tex_file}
% --- end of overleaf setting ---

% mathtools can be used to define labeling format for equations
% one can use \eqref for a reference to a labeled equation.
\usepackage{mathtools}
\newtagform{supp}{(S-}{)}    % define a equation labeling format for suppliment
\usetagform{supp}            % use the supp format
\usetagform{default}         % use the default format

\usepackage{algorithmic}
\usepackage{algorithm}
\renewcommand{\algorithmicrequire}{\textbf{Input:}}
\renewcommand{\algorithmicensure}{\textbf{Output:}}

% package for hyperlinks
% It's error-prone because hyper link is quite difficult
% due to the fact the typesetting environment is complex.
% So you can disable this package and finish the document.
% Then sort out the hyperlink thing.
\usepackage[colorlinks,linkcolor=red,anchorcolor=blue,citecolor=green,CJKbookmarks=True]{hyperref}

% package for displaying highlighted codes
\usepackage{minted}

% package for input codec and output rendering font
\usepackage[utf8]{inputenc}    % it is always good practice to use utf8
                               % you can also try latin1, latin2, cp1252 and cp1250
\usepackage[T1]{fontenc}    % the default is T0, which only contains 128 characters
                            % you can try T1, T2A, T2B
% you can refer to https://www.overleaf.com/learn/latex/international_language_support#Font_encoding
% for more details.


\title{T-test}
\author{Chao Cheng}
\date{\today}



\begin{document}
\maketitle

\section{Basic knowledge}
\label{sec:basic-knowledge}

$\phi\left(x\right)$ and $\Phi\left(x\right)$ are pdf and cdf of standard normal distribution, respectively. We use $Z$ to represent a random variable that follows standard normal distribution and $z_{\alpha}$ the lower $\alpha$ quantile of standard normal distribution. Therefore
\[
  P\left(Z \leq z_{\alpha}\right) = \Phi\left(z_{\alpha}\right) = \alpha
  .
\]

\begin{thm}
  Let $x_1, \cdots, x_n$ be a random sample from a population with mean $\mu$ and variance $\sigma^2 <\infty$. Then
  \begin{enumerate}
  \item $\mathrm{E}\bar{x} = \mu$.
  \item $\mathrm{Var}\bar{x} = \sigma^2 / n$.
  \item $\mathrm{E}S^2 = \sigma^2$, where $S^2 = \frac{1}{n - 1}\sum\limits_{i = 1}^n\left(x_i - \bar{x}\right)^2$.
  \end{enumerate}
\end{thm}

\begin{thm}
  Let $x_1, \cdots, x_n$ be a random sample from $N\left(\mu, \sigma^2\right)$. Then
  \begin{enumerate}
  \item $\bar{X} \sim N\left(\mu, \sigma^2 / n\right)$.
  \item $\bar{X}$ is independent of $S^2$.
  \item $\left(n - 1\right)S^2 / \sigma^2$ follows a chi-squared distribution with $n - 1$ degree of freedom.
  \end{enumerate}
\end{thm}



\section{One-sample test}
\label{sec:one-sample}

Consider a random sample $x_1, \cdots, x_n$ from $N\left(\mu, \sigma^2\right)$. The likelihood is
\[
  \begin{aligned}
    f\left(x_1, \cdots, x_n\right)
    =& \prod\limits_{i = 1}^n
       \left(2\pi\sigma^2\right)^{-1/2}
       \mathrm{exp}\left(
       -\frac{
       \left(x_i - \mu\right)^2
       }{2\sigma^2}
       \right)    \\
    =& \left(2\pi\sigma^2\right)^{-n / 2}
       \mathrm{exp}\left(
       -\frac{\sum\limits_{i = 1}^n\left(x_i - \mu\right)^2}{2\sigma^2}
       \right)
       .
  \end{aligned}
\]
We propose the test
\[
  H_0:\; \mu = \mu_0
  \textbf{\quad v.s\quad}
  H_1:\; \mu \neq \mu_0
\]
\subsection{variance known}
\label{sec:variance-known}

Construct LRT
\[
  LR = \frac{
    \underset{\mu\in H_0}{\mathrm{max}}\;
    f\left(x_1, \cdots, x_n\middle|\mu\right)
  }{
    \underset{\mu\in H_0\cup H_1}{\mathrm{max}}\;
    f\left(x_1, \cdots, x_n\middle|\mu\right)
  }
  =\frac{
    f\left(x_1, \cdots, x_n\middle|\mu = \mu_0\right)
  }{
    f\left(x_1, \cdots, x_n\middle|\mu = \bar{x}\right)
  }
  = \mathrm{exp}\left(
    -\frac{\left(\bar{x} - \mu_0\right)^2}{2\sigma^2 / n}
  \right)
\]

Therefore rejecting $H_0$ when LR is smaller than some constant $C$ is equivalent to rejecting $H_0$ when $\left|\bar{x} - \mu_0\right|$ is larger than some other constant $C$. Hence
\[
  \text{Reject Region: }
  \left\{
    \bar{x}
    :\quad
    \left|\bar{x}-\mu_0\right| > C
  \right\}
\]

\subsubsection{Decide $C$ from $\alpha$}
\label{sec:decide-c-from}

From definition of $\alpha$ we know that $C$ in the reject region is chosen such that
\[
  P\left(
    \left|\bar{x} - \mu_0\right| > C
    \middle| H_0\text{ is true }\right)
  \leq \alpha
  .
\]
But to fully utilize the test, we choose to use equal sign instead of $\leq$. Therefore
\[
  P\left(
    \left|\bar{x} - \mu_0\right| > C
    \middle| \mu = \mu_0
  \right)
  = \alpha
  .
\]
Note that $\bar{x}\sim N\left(\mu, \sigma^2 / n\right)$. Then under the condition $\mu = \mu_0$,
\[
  \frac{\bar{x} - \mu_0}{\sqrt{\sigma^2 / n}}
  \sim N\left(0, 1\right)
  .
\]
Therefore we propose the reject region for $H_0$ being
\[
  \left|
    \frac{\bar{x} - \mu_0}{\sqrt{\sigma^2 / n}}
  \right|
  \geq z_{1 - \alpha / 2}
  .
\]

\subsubsection{Power at given underlying $\mu$}
\label{sec:power-at-given}

The power (the probability to reject $H_0$, when $H_1$ is true) of the proposed test procedure for any given underlying $\mu \neq \mu_0$ is computed as
\[
  \begin{aligned}
    & P\left(
      \left|
      \frac{\bar{x} - \mu_0}{\sqrt{\sigma^2 / n}}
      \right|
      \geq z_{1 - \alpha / 2}
      \right)    \\
    = & P\left(
        \frac{\bar{x} - \mu_0}{\sqrt{\sigma^2 / n}}
        \leq z_{\alpha / 2}
        \right)
        + P\left(
        \frac{\bar{x} - \mu_0}{\sqrt{\sigma^2 / n}}
        \geq z_{1 - \alpha / 2}
        \right)    \\
    = & P\left(
        \frac{\bar{x} - \mu}{\sqrt{\sigma^2 / n}}
        \leq z_{\alpha / 2}
        + \frac{\mu_0 - \mu}{\sqrt{\sigma^2 / n}}
        \right)
        + P\left(
        \frac{\bar{x} - \mu}{\sqrt{\sigma^2 / n}}
        \geq z_{1 - \alpha / 2}
        + \frac{\mu_0 - \mu}{\sqrt{\sigma^2 / n}}
        \right)    \\
    = & P\left(
        Z
        \leq z_{\alpha / 2}
        + \frac{\mu_0 - \mu}{\sqrt{\sigma^2 / n}}
        \right)
        + P\left(
        Z
        \geq z_{1 - \alpha / 2}
        + \frac{\mu_0 - \mu}{\sqrt{\sigma^2 / n}}
        \right)    \\
  \end{aligned}
\]
Here we use the fact that $  \frac{\bar{x} - \mu}{\sqrt{\sigma^2 / n}} \sim N\left(0, 1\right)$. W.l.o.g, assume that $\mu > \mu_0$, then
\[
  P\left(
    Z
    \leq z_{\alpha / 2}
    + \frac{\mu_0 - \mu}{\sqrt{\sigma^2 / n}}
  \right)
\]
would be really close to zero and
\[
  P\left(
    Z
    \geq z_{1 - \alpha / 2}
    + \frac{\mu_0 - \mu}{\sqrt{\sigma^2 / n}}
  \right) 
\]
will offer most of the power. In order to guarantee a power of at least $1 - \beta$, we could simply set
\[
    P\left(
    Z
    \geq z_{1 - \alpha / 2}
    + \frac{\mu_0 - \mu}{\sqrt{\sigma^2 / n}}
  \right)  = 1 - \beta
\]

\subsection{variance unknown}
\label{sec:variance-unknown}


\section{Two sample test}
\label{sec:two-sample-test}


\subsection{Two-sample, variance known}
\label{sec:two-sample-variance}

\subsection{Two-sample, variance unknown but equal}
\label{sec:two-sample-variance-1}

\subsection{Two-sample, variance unknown and unequal}
\label{sec:two-sample-variance-2}







\end{document}



%%% Local Variables:
%%% mode: latex
%%% TeX-master: t
%%% End:
