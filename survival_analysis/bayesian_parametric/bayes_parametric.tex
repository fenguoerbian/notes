\documentclass[a4paper,12pt]{article}
\usepackage{ctex}
\usepackage{geometry}
\geometry{left=2.5cm,right=2.5cm,top=2.5cm,bottom=2.5cm}
\renewcommand{\textfraction}{0.15}
\renewcommand{\topfraction}{0.85}
\renewcommand{\bottomfraction}{0.65}
\renewcommand{\floatpagefraction}{0.60}
\usepackage{amsmath}
\usepackage{amsfonts}
\usepackage{mathrsfs}
\usepackage{amsthm}
\usepackage{extarrows}
\usepackage{bm}
% \newcommand{\bm}{\symbfit}    % `bm` confilicts with `unicode-math`. In that case use \symbfit for bold math symbols
\usepackage{graphicx}
\usepackage[section]{placeins}
\usepackage{flafter}
\usepackage{array}
\usepackage{caption}
\usepackage{subcaption}
\usepackage{color}
\usepackage{multirow}
\usepackage{natbib}
% \usepackage{enumerate}
\usepackage{enumitem}    % more flexible than `enumerate` package, the reference will carry the whole label appearance, not just the counter, unlike the `enumerate` package.
\usepackage{upgreek}    % 'upgreek' letters

% \pdfstringdefDisableCommands{\let\bm=\relax}

\newtheorem{thm}{Theorem}
\newtheorem{cor}{Corollary}
\newtheorem{assum}{Assumption}
\newtheorem{rem}{Remark}
\newtheorem{lem}{Lemma}

\setcounter{topnumber}{5}    % Maximum number of floats that can appear at the top of a text page; default 2. 
\setcounter{bottomnumber}{5}   % Maximum number of floats that can appear at the bottom of a text page; default 1. 
\setcounter{totalnumber}{10}    % Maximum number of floats that can appear on a text page; default 3. 

\DeclareMathOperator*{\argmaxdown}{arg\,max}
\DeclareMathOperator*{\argmindown}{arg\,min}
\DeclareMathOperator{\argmax}{arg\,max}
\DeclareMathOperator{\argmin}{arg\,min}

% cross-reference to other files
% the externaldocument should be compiled 
%   (and at least twice if you're using xr-hyper)
% \usepackage{xr-hyper}
% \usepackage{xr}

% --- external document (ordinary setting) ---
% \externaldocument{external_tex_file}
% --- end of ordinary setting ---

% --- external document (overleaf setting) ---
% externaldocument settings for Overleaf
% \makeatletter
% \newcommand*{\addFileDependency}[1]{% argument=file name and extension
%   \typeout{(#1)}
%   \@addtofilelist{#1}
%   \IfFileExists{#1}{}{\typeout{No file #1.}}
% }
% \makeatother
% \newcommand*{\myexternaldocument}[1]{%
%     \externaldocument{#1}%
%     \addFileDependency{#1.tex}%
%     \addFileDependency{#1.aux}%
% }
% \myexternaldocument{external_tex_file}
% --- end of overleaf setting ---

% mathtools can be used to define labeling format for equations
% one can use \eqref for a reference to a labeled equation.
\usepackage{mathtools}
\newtagform{supp}{(S-}{)}    % define a equation labeling format for suppliment
\usetagform{supp}            % use the supp format
\usetagform{default}         % use the default format

\usepackage{algorithmic}
\usepackage{algorithm}
\renewcommand{\algorithmicrequire}{\textbf{Input:}}
\renewcommand{\algorithmicensure}{\textbf{Output:}}

% package for hyperlinks
% It's error-prone because hyper link is quite difficult
% due to the fact the typesetting environment is complex.
% So you can disable this package and finish the document.
% Then sort out the hyperlink thing.
\usepackage[colorlinks,linkcolor=red,anchorcolor=blue,citecolor=green,CJKbookmarks=True]{hyperref}

% package for displaying highlighted codes
\usepackage{minted}

% package for input codec and output rendering font
\usepackage[utf8]{inputenc}    % it is always good practice to use utf8
                               % you can also try latin1, latin2, cp1252 and cp1250
\usepackage[T1]{fontenc}    % the default is T0, which only contains 128 characters
                            % you can try T1, T2A, T2B
% you can refer to https://www.overleaf.com/learn/latex/international_language_support#Font_encoding
% for more details.

\usepackage{soul}    % \ul{} for underline, \st{} for strikethrough/overstriking, \hl{} for highlingt


\title{Bayesian规则下的Go/Nogo概率}
\author{Chao Cheng}
\date{\today}



\begin{document}
\maketitle

事件时间$T_e\sim Exp(\lambda_1)$, $f(t;\lambda_1) = \lambda_1 \mathrm{exp}\left(-\lambda_1 t\right)$,删失时间$T_c \sim Exp(\lambda_2)$, $f\left(t;\lambda_2\right) = \lambda_2 \mathrm{exp}\left(-\lambda_2t\right)$. 则观察到的数据的联合密度/质量
\[
  f\left(t, \delta\right) = \left\{
    \begin{aligned}
      &\lambda_1\mathrm{exp}\left(-(\lambda_1 + \lambda_2)t\right) && \quad \delta = 1    \\
      &\lambda_2\mathrm{exp}\left(-(\lambda_1 + \lambda_2)t\right) && \quad \delta = 0    \\
    \end{aligned}
  \right.
\]
其中$\delta = 1$表示观察到事件,$\delta = 0$表示观察到删失。那么观察到事件的概率:
\[
  P\left(\delta = 1\right) = \int_0^\infty f\left(t, \delta = 1\right)\mathrm{d}t
  = \frac{\lambda_1}{\lambda_1 + \lambda_2}
  .
\]
那么观察时间的条件分布
\[
  P\left(T \leq t\middle|\delta\right) = \frac{P\left(T\leq t, \delta\right)}{P\left(\delta\right)}
  = \left\{
    \begin{aligned}
          & \frac{P\left(T\leq t, \delta = 1\right)}{P\left(\delta = 1\right)}
    \sim Exp\left(\lambda_1 + \lambda_2\right)
    &&\quad \text{if } \delta = 1    \\
    & \frac{P\left(T\leq t, \delta = 0\right)}{P\left(\delta = 0\right)}
    \sim Exp\left(\lambda_1 + \lambda_2\right)
    &&\quad \text{if } \delta = 0    \\
    \end{aligned}
  \right.
\]
可见该条件分布与$\delta$取值无关,因此$T$与$\delta$独立。(一般这样的结论是不会成立的,这里依赖于两者都是指数分布)。
\par
由此可得
\[
  \begin{aligned}
    \sum\limits_{i = 1}^nT_i \sim Gamma\left(n, \lambda_1 + \lambda_2\right)    \\
    \sum\limits_{i = 1}^n\delta_i \sim Binomial(n, \frac{\lambda_1}{\lambda_1 + \lambda_2})
  \end{aligned}
\]
且$\sum\limits_{i = 1}^nT_i$和$\sum\limits_{i = 1}^n\delta_i$独立。
\par
在Bayesian角度下,风险的后验分布服从$Gamma\left(\text{prior\_shape} + \sum\limits_{i = 1}^n\delta_i, \text{prior\_rate} + \sum\limits_{i = 1}^nT_i\right)$对于给出某一个Go/Nogo的规则,例如$P\left(\lambda < \text{eff\_cut}\right) \geq \text{go\_prob}$后,我们可以将其转化为在所有可能的观察到的事件数的情况下,需要的总观察时间的要求,因此某一个潜在的风险$\lambda_1$给出Go信号的概率为
\[
  \begin{aligned}
    & P\left(\text{Declare Go}\right)    \\
    =& \sum\limits_{evt_{num} = 0}^{n}P\left(
      \text{Declare Go}, \sum\limits_{i = 0}^n\delta_i = evt_{num}
    \right)    \\
    =& \sum\limits_{evt_{num} = 0}^{n}P\left(
      \text{Declare Go}\middle| \sum\limits_{i = 0}^n\delta_i = evt_{num}
    \right)
    P\left(\sum\limits_{i = 0}^n\delta_i = evt_{num}\right)    \\
    =& \sum\limits_{evt_{num} = 0}^{n}
    P\left(
      \sum\limits_{i = 1}^nT_i \geq t_{\text{eff cut for } evt_{num}}
      \middle| \sum\limits_{i = 0}^n\delta_i = evt_{num}
    \right)
    P\left(\sum\limits_{i = 0}^n\delta_i = evt_{num}\right)    \\
    =& \sum\limits_{evt_{num} = 0}^{n}
    P\left(
      \sum\limits_{i = 1}^nT_i \geq t_{\text{eff cut for } evt_{num}}
    \right)
    P\left(\sum\limits_{i = 0}^n\delta_i = evt_{num}\right)
    .
  \end{aligned}
\]
同理我们可以算出某个$\lambda_1$给出Nogo信号的概率为
\[
  P\left(\text{Declare Nogo}\right)
  = \sum\limits_{evt_{num} = 0}^{n}
    P\left(
      \sum\limits_{i = 1}^nT_i \leq t_{\text{fut cut for } evt_{num}}
    \right)
    P\left(\sum\limits_{i = 0}^n\delta_i = evt_{num}\right)
    .
\]






\clearpage
\appendix


\end{document}


%%%%%%%%%%%%%%%%%%%%%Subfigure
\begin{figure}[htb]
\centering
\begin{subfigure}[h]{0.49\textwidth}
\centering
\includegraphics[width=\textwidth]{5_23contour}
\caption{Contour Graphic}
\end{subfigure}
\begin{subfigure}[h]{0.49\textwidth}
\centering
\includegraphics[width=\textwidth]{5_23lattice}
\caption{Lattice Graphic}
\end{subfigure}
\caption{\label{fig:5.23.contour}}
\end{figure}

%%%%%%%%%%%%%%%%%%%%%TukeyHSD
\par
\makeatletter\def\@captype{figure}\makeatother
\begin{minipage}{.45\textwidth}
  \begin{figure}
    \centering
    \includegraphics[width=\textwidth]{5_10tukeyplot}
    \caption{Comparing Yield according to Pressure}
    \label{fig:5.10.tukeyplot}
  \end{figure}
\end{minipage}
\makeatletter\def\@captype{table}\makeatother
\begin{minipage}{.45\textwidth}
  \begin{table}
    \centering
    \begin{tabular}{rrrrr}
      \hline
      & diff & lwr & upr & p adj \\
      \hline
      215-200 & 0.32 & 0.11 & 0.52 & 0.00 \\
      230-200 & -0.18 & -0.39 & 0.02 & 0.08 \\
      230-215 & -0.50 & -0.70 & -0.30 & 0.00 \\
      \hline
    \end{tabular}
    \caption{Tukey multiple comparisons}
    \label{tab:5.10.tukeytable}
  \end{table}
\end{minipage}



%%%%%%%%%%%%%%%%%%%%%%%%custimize \item
\newcounter{Lcount}
\setcounter{Lcount}{0}
\begin{list}{2.1.\arabic{Lcount}}{\usecounter{Lcount}}
\item
\arabic 1, 2, 3 ...
\alph a, b, c ...
\Alph A, B, C ...
\roman i, ii, iii ...
\Roman I, II, III ...
\fnsymbol 星号,单剑号,双剑号等


\end{list}


%%% Local Variables:
%%% mode: latex
%%% TeX-master: t
%%% End:
