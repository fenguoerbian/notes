\documentclass[a4paper,12pt]{article}
\usepackage{geometry}
\geometry{left=2.5cm,right=2.5cm,top=2.5cm,bottom=2.5cm}
\renewcommand{\textfraction}{0.15}
\renewcommand{\topfraction}{0.85}
\renewcommand{\bottomfraction}{0.65}
\renewcommand{\floatpagefraction}{0.60}
\usepackage{amsmath}
\usepackage{amsfonts}
\usepackage{mathrsfs}
\usepackage{amsthm}
\usepackage{extarrows}
\usepackage{bm}
% \newcommand{\bm}{\symbfit}    % `bm` confilicts with `unicode-math`. In that case use \symbfit for bold math symbols
\usepackage{graphicx}
\usepackage[section]{placeins}
\usepackage{flafter}
\usepackage{array}
\usepackage{caption}
\usepackage{subcaption}
\usepackage{color}
\usepackage{multirow}
\usepackage{natbib}
% \usepackage{enumerate}
\usepackage{enumitem}    % more flexible than `enumerate` package, the reference will carry the whole label appearance, not just the counter, unlike the `enumerate` package.
\usepackage{upgreek}    % 'upgreek' letters

% \pdfstringdefDisableCommands{\let\bm=\relax}

\newtheorem{thm}{Theorem}
\newtheorem{cor}{Corollary}
\newtheorem{assum}{Assumption}
\newtheorem{rem}{Remark}
\newtheorem{lem}{Lemma}
\newtheorem{prop}{Proposition}

\setcounter{topnumber}{5}    % Maximum number of floats that can appear at the top of a text page; default 2. 
\setcounter{bottomnumber}{5}   % Maximum number of floats that can appear at the bottom of a text page; default 1. 
\setcounter{totalnumber}{10}    % Maximum number of floats that can appear on a text page; default 3. 

\DeclareMathOperator*{\argmaxdown}{arg\,max}
\DeclareMathOperator*{\argmindown}{arg\,min}
\DeclareMathOperator{\argmax}{arg\,max}
\DeclareMathOperator{\argmin}{arg\,min}

% cross-reference to other files
% the externaldocument should be compiled 
% (and at least twice if you're using xr-hyper)
% \usepackage{xr-hyper}
% \usepackage{xr}

% --- external document (ordinary setting) ---
% \externaldocument{external_tex_file}
% --- end of ordinary setting ---

% --- external document (overleaf setting) ---
% externaldocument settings for Overleaf
% \makeatletter
% \newcommand*{\addFileDependency}[1]{% argument=file name and extension
% \typeout{(#1)}
% \@addtofilelist{#1}
% \IfFileExists{#1}{}{\typeout{No file #1.}}
% }
%   \makeatother
%   \newcommand*{\myexternaldocument}[1]{%
%   \externaldocument{#1}%
%   \addFileDependency{#1.tex}%
%   \addFileDependency{#1.aux}%
% }
%   \myexternaldocument{external_tex_file}
%   --- end of overleaf setting ---

%   mathtools can be used to define labeling format for equations
%   one can use \eqref for a reference to a labeled equation.
\usepackage{mathtools}
\newtagform{supp}{(S-}{)}    % define a equation labeling format for suppliment
\usetagform{supp}            % use the supp format
\usetagform{default}         % use the default format

\usepackage{algorithmic}
\usepackage{algorithm}
\renewcommand{\algorithmicrequire}{\textbf{Input:}}
\renewcommand{\algorithmicensure}{\textbf{Output:}}

% package for hyperlinks
% It's error-prone because hyper link is quite difficult
% due to the fact the typesetting environment is complex.
% So you can disable this package and finish the document.
% Then sort out the hyperlink thing.
\usepackage[colorlinks,linkcolor=red,anchorcolor=blue,citecolor=green,CJKbookmarks=True]{hyperref}

% package for displaying highlighted codes
\usepackage{minted}

% package for input codec and output rendering font
\usepackage[utf8]{inputenc}    % it is always good practice to use utf8
% you can also try latin1, latin2, cp1252 and cp1250
\usepackage[T1]{fontenc}    % the default is T0, which only contains 128 characters
% you can try T1, T2A, T2B
% you can refer to https://www.overleaf.com/learn/latex/international_language_support#Font_encoding
% for more details.


\title{Logrank Test}
\author{Chao Cheng}
\date{\today}



\begin{document}
\maketitle
\tableofcontents{}

\section{Introduction}
\label{sec:introduction}

The log-rank test is one of the most commonly used test for comparing two or more survival distributions. To simplify the discussion, let's assume there are two groups of subjects, coded by 0 and 1. In group $j$, there are $n_j$ i.i.d. underlying survival times with common c.d.f. denoted by $F_j\left(\cdot\right)$. And the corresponding hazard, cummulative hazard and survival functions are denoted by $h_j\left(\cdot\right)$, $H_j\left(\cdot\right)$ and $S_j\left(\cdot\right)$, respectively.
\par
As usual, we assume the {\color{red} non-informative right censoring}. So in each group, $T_i$ and $C_i$ are independent.
\par
Here we want to test the null hypothesis $F_1\left(\cdot\right) = F_2\left(\cdot\right)$. If we know the parametric form of $F_1\left(\cdot\right)$ and $F_2\left(\cdot\right)$, e.g. the exponential distribution family, then this test can be reduced to test against a point/region in a Eucilidean parameter space. However, here we want a non-parametric test; that is, a test whose validity dose not depend on any parametric assumptions.
\par
Clearly, a UMP test can not exist for this type of hypothesis. And there are two options in this case:
\begin{itemize}
\item \textbf{Directional test: }These are oriented towards a specific type of difference, e.g. $S_1\left(t\right) = S_0\left(t\right)^\theta$ for some $\theta$.
\item \textbf{Omnibus test: }These test are designed to have some power against all types of difference, e.g. a test based on $\int\left|S_1\left(t\right) - S_0\left(t\right)\right|\mathrm{d}t$ over some time interval.
\end{itemize}

\begin{table}[htbp]
  \centering
  \begin{tabular}{|r|p{4cm}|p{4cm}|}
    \hline
    & Pros & Cons    \\
    \hline
    Directional test &
                       Strong power against the specified type of difference &
    (often) poor power against other types of difference    \\
    \hline
    Omnibus test & have some power against most types of difference &
    lower power compared to a directional test for certain types of difference \\
    \hline
  \end{tabular}
  \caption{Pros and cons for different types of tests}
  \label{tab:pros_and_cons}
\end{table}

The Pros-and-Cons of these two options of tests are summarised in Table~\ref{tab:pros_and_cons}. And a chioce between these two types of tests in real application involves several factors. Here we just point out that log-rank test is a directional test, and the specific type is the {\color{red} constant hazard ratio over time}.

\section{Log-rank test}
\label{sec:log-rank-test}

Log-rank test can be viewed as modification for the contingency table test to allow censoring in the data. Now let's consider these 2 groups, and denote the \underline{distinct} times of \underline{observed} failures as $0 < \tau_1 < \cdots < \tau_k$. We also define
\[
  \begin{aligned}
    Y_i\left(\tau_j\right) =&\;\text{number at risk (including events) for group $i$ at $\tau_j$}     \\
    Y\left(\tau_j\right) =& Y_0\left(\tau_j\right) + Y_1\left(\tau_j\right)    \\
    d_{ij} =&\;\text{number of events for group $i$ at $\tau_j$}   \\
    d_j =& d_{0j} + d_{1j}
  \end{aligned}
\]
Then the information at time $\tau_j$ can be summarized in the following $2\times 2$ table(Table~\ref{tab:contingency_table_at_tau_j}):
\begin{table}[htbp]
  \centering
  \begin{tabular}{|r|c|c|c|}
    \hline
    Group & event & no event & number at risk    \\
    \hline
    Group 0 & $d_{0j}$ & $Y_0\left(\tau_j\right) - d_{0j}$ & $Y_0\left(\tau_j\right)$    \\
    \hline
    Group 1 & $d_{1j}$ & $Y_1\left(\tau_j\right) - d_{1j}$ & $Y_1\left(\tau_j\right)$    \\
    \hline
    Overall & $d_j$ & $Y\left(\tau_j\right) - d_{j}$ & $Y\left(\tau_j\right)$    \\
    \hline
  \end{tabular}
  \caption{Information at $\tau_j$}
  \label{tab:contingency_table_at_tau_j}
\end{table}

Note that $d_{0j}/ Y_{0}\left(\tau_j\right)$, $d_{1j}/ Y_{1}\left(\tau_j\right)$ and $d_{j}/ Y\left(\tau_j\right)$ are the estimates of $h_0\left(\tau_j\right)$, $h_1\left(\tau_j\right)$ and $h\left(\tau_j\right)$. To test the difference between $F_0\left(\cdot\right)$ and $F_1\left(\cdot\right)$ at this time point $\tau_j$, one can consider the $\chi^2$-test ({\color{blue} details of $\chi^2$-test can be found in other notes}). But here we use the Fisher exact test, which is conditional on the marginal counts $Y_0\left(\tau_j\right)$, $Y_1\left(\tau_j\right)$, $d_j$ and $Y\left(\tau_j\right) - d_j$. ({\color{blue} This is more suitable in survival scenario because we know that the estimates are always conditional on the previous results. And this is just my personal opinion.})
\par
Now, given those four marginal counts and $H_0: F_0\left(\cdot\right) = F_1\left(\cdot\right)$, one can see that $d_{1j}$ determines the whole table and actually $d_{1j}$ follows a hypergeometric distribution
\[
  P\left(D_{1j} = d\right) =
  \frac{
    C_{Y_0\left(\tau_j\right)}^{d_{0j}}
    C_{Y_1\left(\tau_j\right)}^{d_j - d_{0j}}
  }{
    C_{Y\left(\tau_j\right)}^{d_j}
  }
  ,
\]
where $d$ ranges such that
\[
  \begin{aligned}
    d &\geq 0    \\
    d_j - d &\geq 0    \\
    Y_1\left(\tau_j\right) - d &\geq 0    \\
    Y_0\left(\tau_j\right) - \left(d_j - d\right) &\geq 0
  \end{aligned}
\]
Therefore
\[
  \mathrm{max}\left(0, d_j - Y_0\left(\tau_j\right)\right)
  \leq d
  \leq \mathrm{min}\left(d_j, Y_1\left(\tau_j\right)\right)
  .
\]
And it's easy to know that
\[
  \begin{aligned}
    \mathrm{E}_j &= \mathrm{E}\left(D_{1j}\right) &&= \frac{Y_1\left(\tau_j\right)d_j}{Y\left(\tau_j\right)}    \\
    \mathrm{V}_j &= \mathrm{Var}\left(D_{1j}\right) &&=
                                       \frac{Y\left(\tau_j\right) - Y_1\left(\tau_j\right)}{Y\left(\tau_j\right) - 1}
                                       \cdot Y_1\left(\tau_j\right)
                                       \left(
                                       \frac{d_j}{Y\left(\tau_j\right)}
                                       \right)
                                       \left(
                                       1 - \frac{d_j}{Y\left(\tau_j\right)}
                                       \right)    \\
    && &= \frac{
       Y_0\left(\tau_j\right)Y_1\left(\tau_j\right)d_j\left(Y\left(\tau_j\right) - d_j\right)
       }{
       Y\left(\tau_j\right)^2
       \left(Y\left(\tau_j\right) - 1\right)
       }
  \end{aligned}
\]
And denote the observation $O_j = d_{1j}$. And we can define for over the whole time points
\[
  \begin{aligned}
    & \mathrm{O} = \sum\limits_{j = 1}^k O_j    \\
    & \mathrm{E} = \sum\limits_{j = 1}^k \mathrm{E}_j    \\
    & \mathrm{V} = \sum\limits_{j = 1}^k \mathrm{V}_j
  \end{aligned}
\]
And the test statistic is argued to follow under $H_0$:
\[
  Z = \frac{O - \mathrm{E}}{\sqrt{\mathrm{V}}}
  \overset{apx}{\sim}\; N\left(0, 1\right).
\]
Some comments about this test are
\begin{itemize}
\item $E_j$ can be viewed as a conditional expectation for each $j$, but strictly speaking, it is not clear that $E$ has such an interpretation. 
\item The construction of $Z$ seems to imply that the distributions from each $\tau_j$ are independent $N\left(0, 1\right)$ under $H_0$. Is this true/accurate?
\item Note that $Y_0\left(\tau_1\right), Y_0\left(\tau_2\right), \cdots$ and $Y_1\left(\tau_1\right), Y_1\left(\tau_2\right), \cdots$ are nonincreasing, and as soon as one reaches 0, it \textbf{must} follow that $O_j = E_j$ and $V_j = 0$ \textbf{at and beyond} that time. Thus there is no contribution for the data after that time point when computing $Z$.
\item The logrank test is a \textbf{directional} test oriented towards alternatives where $S_1\left(t\right) = \left(S_0\left(t\right)\right)^\theta$, or equivalently, when $h_1\left(t\right) / h_0\left(t\right) = \theta$, the constant hazard ratio alternative.
\item The logrank statistic arise as a score test from a partial likelihood function from Cox's proportional hazards model.
\item This construction of logrank test in this notes is an intuitive way. Some more serious technical details, including how does the test behave as a function of the amount of censoring or the hazard functions, will be discussed in other notes.
\end{itemize}


\section{Extensions of Log-rank test}
\label{sec:extensions-log-rank}

\subsection{Stratified logrank test}
\label{sec:strat-logr-test}

For example in this case, we want to take into account (adjust for) some covariates, such as gender, in addition to the treatment-control (group1-group0) group. Then actually we will have 4 groups of subjects. In general, if we have overall $L$ stratified levels ($L$ is often the product of levels from each stratified covariates), then we want to test
\[
  H_0:\; S_0^{\left(l\right)}\left(\cdot\right) = S_1^{\left(l\right)}\left(\cdot\right)
  ,\quad l = 1, \cdots, L.
\]
\par
This stratified test is useful when the distributions of the stratum variable in the treatment-control groups are different, but the distribution of these relavent covariates within each stratum is the same between the treatment-control groups. And the test can be constructed as follow:

\begin{enumerate}
\item Seperate data into $L$ groups according to your strata variables.
\item Compute $O^{\left(l\right)}$, $E^{\left(l\right)}$ and $V^{\left(l\right)}$ within each stratified level, just as with ordinary logrank test.
\item Compute the test statistics
  \[
    Z = \frac{\sum\limits_{l = 1}^L\left(O^{\left(l\right)} - E^{\left(l\right)}\right)}{
      \sqrt{\sum\limits_{l = 1}^L V^{\left(l\right)}}
    }.
  \]
  and under $H_0$, we have
  \[
    Z \overset{apx}{\sim} N\left(0, 1\right)
    .
  \]
\end{enumerate}

Some comments about the stratified logrank tests:
\begin{itemize}
\item Intuitively, I think this test assumes $\left\{O^{\left(l\right)}, E^{\left(l\right)}, V^{\left(l\right)}\right\}$ are (approximately) uncorrelated, so the test statistic will be $N\left(0, 1\right)$.
\item If there are too many strata, this test will have poor power. That's because there is no contribution for any $2\times2$ table once one of the $Y_l\left(\tau_j\right)$ becomes zero.
\item The stratified logrank test also arises as a score test from Cox's model. This relationship will also clarify the types of alternatives to $H_0$ for which the stratified logrank test is directed.
\end{itemize}

\subsection{Weighted logrank test}
\label{sec:weight-logr-test}

Note that in logrank test, $O_j - E_j$ measures of how $h_0\left(\tau_j\right)$ and $h_1\left(\tau_j\right)$ differ. And in ordinary logrank test, each time point has the contribution (weights) to the final test statistic.
\par
Suppose we want to compare groups, but in a way that \textbf{emphasizes} certain times more than others. Let $w_1, w_2, \cdots, w_k$ be known, non-negative constants. The weighted logrank test is given by
\[
  Z_w = \frac{\sum\limits_{j = 1}^Kw_j\left(O_j - E_j\right)}{
    \sqrt{\sum\limits_{j = 1}^Kw_j^2V_j}
  }
  ,
\]
and under $H_0$, $Z_w \overset{apx}{\sim}N\left(0, 1\right)$.
\par
Some comments about the weighted logrank test:
\begin{itemize}
\item Choosing $w_i = 1$ for all $i$ is just the ordinary logrank test.
\item One can choose to place larger weights at those $\tau_j$s where larger difference is anticipated. But what dose ``difference'' refer to? $h_0\left(\tau_j\right) - h_1\left(\tau_j\right)$, $h_0\left(\tau_j\right) / h_1\left(\tau_j\right)$, $S_0\left(\tau_j\right)/S_1\left(\tau_j\right)$, $\cdots$
\item Choosing $w_j = Y\left(\tau_j\right)$ yields the \textbf{generalized wilcoxon} test. Since $Y\left(\tau_1\right) > Y\left(\tau_2\right) > \cdots$, this generalized wilcoxon test places greater weights on \textbf{early} differences between $h_0\left(\cdot\right)$ and $h_1\left(\cdot\right)$.
\end{itemize}

\subsection{Logrank test for multiple groups}
\label{sec:logr-test-mult}

So now we want to compare the survival functions among several($>2$) groups. Specifically, there are $p+1$ groups, indexed by $0, 1, 2, \cdots, p$ and the hypothesis is
\[
  H_0:\;
  S_0\left(\cdot\right) = S_1\left(\cdot\right) = \cdots = S_p\left(\cdot\right)
\]
Then an extension of the ordinary logrank test can be constructed from Table~\Ref{tab:multiple_groups_contingency_table_at_tau_j}:

\begin{table}[htbp]
  \centering
  \begin{tabular}{|r|c|c|c|}
    \hline
    Group & event & no event & number at risk    \\
    \hline
    Group 0 & $d_{0j}$ & $Y_0\left(\tau_j\right) - d_{0j}$ & $Y_0\left(\tau_j\right)$    \\
    \hline
    Group 1 & $d_{1j}$ & $Y_1\left(\tau_j\right) - d_{1j}$ & $Y_1\left(\tau_j\right)$    \\
    \hline
    $\vdots$ & $\vdots$ & $\vdots$ & $\vdots$     \\
    \hline
    Group $p$ & $d_{pj}$ & $Y_j\left(\tau_j\right) - d_{pj}$ & $Y_p\left(\tau_j\right)$    \\
    \hline
    Overall & $d_j = \sum\limits_{i = 0}^pd_{ij}$
                  & $Y\left(\tau_j\right) - d_{j}$
                             & $Y\left(\tau_j\right) = \sum\limits_{i = 0}^pY_i\left(\tau_j\right)$    \\
    \hline
  \end{tabular}
  \caption{Information at $\tau_j$ for multiple groups}
  \label{tab:multiple_groups_contingency_table_at_tau_j}
\end{table}
And like before, we can construct
\[
  \bm{O}_j =
  \begin{pmatrix}
    d_{1j}    \\
    d_{2j}    \\
    \vdots    \\
    d_{pj}
  \end{pmatrix}_{p\times 1}
  ,\quad
  \bm{E}_j =
  \begin{pmatrix}
    E_{1j}    \\
    E_{2j}    \\
    \vdots    \\
    E_{pj}
  \end{pmatrix}_{p\times 1}
  ,\quad\text{where }
  E_{ij} = \frac{Y_i\left(\tau_j\right)}{Y\left(\tau_j\right)}\cdot d_j
\]
and
\[
  \bm{V}_j =
  \begin{pmatrix}
    V_{11,j} & \cdots & V_{1p,j}    \\
    \vdots & \ddots & \vdots    \\
    V_{p1,j} & \cdots & V_{pp,j}
  \end{pmatrix}_{p\times p}
  ,\quad\text{where }
  V_{kl,j} = \frac{
    d_j\left(Y\left(\tau_j\right) - d_j\right)Y_k\left(\tau_j\right)
    \left(
      Y\left(\tau_j\right) \cdot 1\left(k = l\right)
      - Y_k\left(\tau_j\right)
    \right)
  }{
    Y\left(\tau_j\right)^2\left(Y\left(\tau_j\right) - 1\right)
  }
\]
Then with
\[
  \bm{O} = \sum\limits_{j = 1}^K\bm{O}_j
  ,\quad
  \bm{E} = \sum\limits_{j = 1}^K\bm{E}_j
  ,\quad
  \bm{V} = \sum\limits_{j = 1}^K\bm{V}_j
\]
we have under $H_0$:
\[
  Q_p  = \left(
    \bm{O} - \bm{E}
  \right)^T
  \bm{V}^{-1}
  \left(\bm{O} - \bm{E}\right)
  \overset{apx}{\sim}
  \chi^2_p
  .
\]
Some comments about this test
\begin{itemize}
\item This test is \textbf{omnibus} in terms of how it combines the $p+1$ groups; i.e., it is not directed towards a dose-response, in contrast to the trend test below.
\end{itemize}


\subsection{Logrank trend test}
\label{sec:logrank-trend-test}


How can we test for a \textbf{trend} in the survival functions in the $p+1$ groups? For example, we want to test for a dose-response effect, where higher dose usage implies more significant response (the risk of failure to be monotone with exposure/dose). So we want to design a test that is especially oriented towards this type of \underline{alternative} to $H_0$.
\par
Let $\bm{c} = $ any $p\times 1$ vector of constants. If $\bm{O} - \bm{E} \overset{apx}{\sim} N\left(\bm{0}, \bm{V}\right)$ under $H_0$, then
\[
  \bm{c}^T\left(\bm{O} - \bm{E}\right)
  \overset{apx}{\sim}
  N\left(\bm{0}, \bm{c}^T\bm{V}\bm{c}\right)
  ,
\]
which means
\[
  Z_{tr} = \frac{
    \bm{c}^T\left(\bm{O} - \bm{E}\right)
  }{
    \sqrt{\bm{c}^T\bm{V}\bm{c}}
  }
  \overset{apx}{\sim}
  N\left(0, 1\right)
  ,\quad\text{under }H_0.
\]
Some comments are
\begin{itemize}
\item How to choose $\bm{c}$? One might consider setting $c_j = D_j$ for $j = 1, \cdots p$ where $D_j$ is the dose usage for group $j$. But in summary the choice of $\bm{c}$ depends in part on the setting. Sometimes we want to test for monotone trend, other times we want to distinguish a linear versus superlinear(e.g., quadratic) dose-response. 
\item Entries in $\bm{c}$ should be \textbf{monotone}, but against what specific alternative is a paticular choice of $\bm{c}$ optimal and what are the consequences of selecting the \underline{wrong} value of $\bm{c}$?
\end{itemize}

\section{Power analysis}
\label{sec:power-analysis}

The power of logrank test under alternative $H_1: h_1\left(t\right) = h_0\left(t\right)e^\beta$ is approximately
\[
  \Phi\left(
    \left|\beta\right|
    \sqrt{D\pi_0\left(1 - \pi_0\right)}
    - 1.96
  \right)
  ,
\]
where $D$ is the expected number of failures and $\pi_0$ is the proportion of patient in groups 0.

{\color{red}To be added.}
The weighted logrank test, $Z_w$, follows $N\left(\theta, 1\right)$, where $\theta = \xi / \sigma_w$, where
\begin{equation}
  \label{eq:xi}
  \xi = \sqrt{p\left(1 - p\right)}
  \int_0^{\infty}
  {\color{red}\text{to be added.}}
\end{equation}
and
\begin{equation}
  \label{eq:sigma}
  \sigma_w^2 =
  \int_0^\infty
  {\color{red}\text{to be added.}}
\end{equation}
In later analysis, we assume  $P\left(z_i = 1\right) = p$ for all $i$. And
\[
  p\left(t\right) = P\left(Z_i = 1\middle|U_i\geq t\right)
  .
\]
\par
While for parametric test for exponential data, the NCP $\theta_{exp}$ is
\[
  \theta^2_{exp}
  = \frac{
    p\left(1 - p\right)d_0d_1
  }{
    pd_1 + \left(1 - p\right)d_0
  }
  ,
\]
where $d_j = P\left(\delta_i = 1\middle|z_i = j, H_0\right) = \int_0^{\infty}f_0\left(t\right)\left(1 - G_j\left(t\right)\right)\mathrm{d}t$, $j = 0, 1$. 
\par

\paragraph{$G_0\left(\cdot\right) = G_1\left(\cdot\right)$}: In the censoring distribution in the 2 groups are identical, $d_0 = d_1$ and $p\left(t\right) = p$. Therefore
\[
  ARE = \frac{\theta_{LR}}{\theta_{exp}} = 1
  ,
\]
where $\theta_{LR}$ is the NCP for logrank test.
\par
\textbf{Note:} this NCP takes different form than the previous approximated one for sample size calculation.




\bibliographystyle{plainnat}
\bibliography{../ref}





\end{document}


%%% Local Variables:
%%% mode: latex
%%% TeX-master: t
%%% End:
