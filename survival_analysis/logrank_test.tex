\documentclass[a4paper,12pt]{article}
\usepackage{geometry}
\geometry{left=2.5cm,right=2.5cm,top=2.5cm,bottom=2.5cm}
\renewcommand{\textfraction}{0.15}
\renewcommand{\topfraction}{0.85}
\renewcommand{\bottomfraction}{0.65}
\renewcommand{\floatpagefraction}{0.60}
\usepackage{amsmath}
\usepackage{amsfonts}
\usepackage{mathrsfs}
\usepackage{amsthm}
\usepackage{extarrows}
\usepackage{bm}
% \newcommand{\bm}{\symbfit}    % `bm` confilicts with `unicode-math`. In that case use \symbfit for bold math symbols
\usepackage{graphicx}
\usepackage[section]{placeins}
\usepackage{flafter}
\usepackage{array}
\usepackage{caption}
\usepackage{subcaption}
\usepackage{color}
\usepackage{multirow}
\usepackage{natbib}
% \usepackage{enumerate}
\usepackage{enumitem}    % more flexible than `enumerate` package, the reference will carry the whole label appearance, not just the counter, unlike the `enumerate` package.
\usepackage{upgreek}    % 'upgreek' letters

% \pdfstringdefDisableCommands{\let\bm=\relax}

\newtheorem{thm}{Theorem}
\newtheorem{cor}{Corollary}
\newtheorem{assum}{Assumption}
\newtheorem{rem}{Remark}
\newtheorem{lem}{Lemma}
\newtheorem{prop}{Proposition}

\setcounter{topnumber}{5}    % Maximum number of floats that can appear at the top of a text page; default 2. 
\setcounter{bottomnumber}{5}   % Maximum number of floats that can appear at the bottom of a text page; default 1. 
\setcounter{totalnumber}{10}    % Maximum number of floats that can appear on a text page; default 3. 

\DeclareMathOperator*{\argmaxdown}{arg\,max}
\DeclareMathOperator*{\argmindown}{arg\,min}
\DeclareMathOperator{\argmax}{arg\,max}
\DeclareMathOperator{\argmin}{arg\,min}

% cross-reference to other files
% the externaldocument should be compiled 
% (and at least twice if you're using xr-hyper)
% \usepackage{xr-hyper}
% \usepackage{xr}

% --- external document (ordinary setting) ---
% \externaldocument{external_tex_file}
% --- end of ordinary setting ---

% --- external document (overleaf setting) ---
% externaldocument settings for Overleaf
% \makeatletter
% \newcommand*{\addFileDependency}[1]{% argument=file name and extension
% \typeout{(#1)}
% \@addtofilelist{#1}
% \IfFileExists{#1}{}{\typeout{No file #1.}}
% }
%   \makeatother
%   \newcommand*{\myexternaldocument}[1]{%
%   \externaldocument{#1}%
%   \addFileDependency{#1.tex}%
%   \addFileDependency{#1.aux}%
% }
%   \myexternaldocument{external_tex_file}
%   --- end of overleaf setting ---

%   mathtools can be used to define labeling format for equations
%   one can use \eqref for a reference to a labeled equation.
\usepackage{mathtools}
\newtagform{supp}{(S-}{)}    % define a equation labeling format for suppliment
\usetagform{supp}            % use the supp format
\usetagform{default}         % use the default format

\usepackage{algorithmic}
\usepackage{algorithm}
\renewcommand{\algorithmicrequire}{\textbf{Input:}}
\renewcommand{\algorithmicensure}{\textbf{Output:}}

% package for hyperlinks
% It's error-prone because hyper link is quite difficult
% due to the fact the typesetting environment is complex.
% So you can disable this package and finish the document.
% Then sort out the hyperlink thing.
\usepackage[colorlinks,linkcolor=red,anchorcolor=blue,citecolor=green,CJKbookmarks=True]{hyperref}

% package for displaying highlighted codes
\usepackage{minted}

% package for input codec and output rendering font
\usepackage[utf8]{inputenc}    % it is always good practice to use utf8
% you can also try latin1, latin2, cp1252 and cp1250
\usepackage[T1]{fontenc}    % the default is T0, which only contains 128 characters
% you can try T1, T2A, T2B
% you can refer to https://www.overleaf.com/learn/latex/international_language_support#Font_encoding
% for more details.


\title{Logrank Test}
\author{Chao Cheng}
\date{\today}



\begin{document}
\maketitle
\tableofcontents{}

\section{Introduction}
\label{sec:introduction}

The log-rank test is one of the most commonly used test for comparing two or more survival distributions. To simplify the discussion, let's assume there are two groups of subjects, coded by 0 and 1. In group $j$, there are $n_j$ i.i.d. underlying survival times with common c.d.f. denoted by $F_j\left(\cdot\right)$. And the corresponding hazard, cummulative hazard and survival functions are denoted by $h_j\left(\cdot\right)$, $H_j\left(\cdot\right)$ and $S_j\left(\cdot\right)$, respectively.
\par
As usual, we assume the {\color{red} non-informative right censoring}. So in each group, $T_i$ and $C_i$ are independent.
\par
Here we want to test the null hypothesis $F_1\left(\cdot\right) = F_2\left(\cdot\right)$. If we know the parametric form of $F_1\left(\cdot\right)$ and $F_2\left(\cdot\right)$, e.g. the exponential distribution family, then this test can be reduced to test against a point/region in a Eucilidean parameter space. However, here we want a non-parametric test; that is, a test whose validity dose not depend on any parametric assumptions.
\par
Clearly, a UMP test can not exist for this type of hypothesis. And there are two options in this case:
\begin{itemize}
\item \textbf{Directional test: }These are oriented towards a specific type of difference, e.g. $S_1\left(t\right) = S_0\left(t\right)^\theta$ for some $\theta$.
\item \textbf{Omnibus test: }These test are designed to have some power against all types of difference, e.g. a test based on $\int\left|S_1\left(t\right) - S_0\left(t\right)\right|\mathrm{d}t$ over some time interval.
\end{itemize}

\begin{table}[htbp]
  \centering
  \begin{tabular}{|r|p{4cm}|p{4cm}|}
    \hline
    & Pros & Cons    \\
    \hline
    Directional test &
                       Strong power against the specified type of difference &
    (often) poor power against other types of difference    \\
    \hline
    Omnibus test & have some power against most types of difference &
    lower power compared to a directional test for certain types of difference \\
    \hline
  \end{tabular}
  \caption{Pros and cons for different types of tests}
  \label{tab:pros_and_cons}
\end{table}

The Pros-and-Cons of these two options of tests are summarised in Table~\ref{tab:pros_and_cons}. And a chioce between these two types of tests in real application involves several factors. Here we just point out that log-rank test is a directional test, and the specific type is the {\color{red} constant hazard ratio over time}.

\section{Log-rank test}
\label{sec:log-rank-test}

Log-rank test can be viewed as modification for the contingency table test to allow censoring in the data. Now let's consider these 2 groups, and denote the \underline{distinct} times of \underline{observed} failures as $0 < \tau_1 < \cdots < \tau_k$. We also define
\[
  \begin{aligned}
    Y_i\left(\tau_j\right) =&     \\
    Y\left(\tau_j\right) =& Y_0\left(\tau_j\right) + Y_1\left(\tau_j\right)    \\
    d_{ij} =&    \\
    d_j =& d_{0j} + d_{1j}
  \end{aligned}
\]
Then the information at time $\tau_j$ can be summarized in the following $2\times 2$ table(Table~\ref{tab:contingency_table_at_tau_j}):
\begin{table}[htbp]
  \centering
  \begin{tabular}{|r|c|c|c|}
    \hline
    Group & event & no event & number at risk    \\
    \hline
    Group 0 & $d_{0j}$ & $Y_0\left(\tau_j\right) - d_{0j}$ & $Y_0\left(\tau_j\right)$    \\
    \hline
    Group 1 & $d_{1j}$ & $Y_1\left(\tau_j\right) - d_{1j}$ & $Y_1\left(\tau_j\right)$    \\
    \hline
    Overall & $d_j$ & $Y\left(\tau_j\right) - d_{j}$ & $Y\left(\tau_j\right)$    \\
    \hline
  \end{tabular}
  \caption{Information at $\tau_j$}
  \label{tab:contingency_table_at_tau_j}
\end{table}

Note that $d_{0j}/ Y_{0}\left(\tau_j\right)$, $d_{1j}/ Y_{1}\left(\tau_j\right)$ and $d_{j}/ Y\left(\tau_j\right)$ are the estimates of $h_0\left(\tau_j\right)$, $h_1\left(\tau_j\right)$ and $h\left(\tau_j\right)$. To test the difference between $F_0\left(\cdot\right)$ and $F_1\left(\cdot\right)$ at this time point $\tau_j$, one can consider the $\chi^2$-test ({\color{blue} details of $\chi^2$-test can be found in other notes}). But here we use the Fisher exact test, which is conditional on the marginal counts $Y_0\left(\tau_j\right)$, $Y_1\left(\tau_j\right)$, $d_j$ and $Y\left(\tau_j\right) - d_j$. ({\color{blue} This is more suitable in survival scenario because we know that the estimates are always conditional on the previous results. And this is just my personal opinion.})
\par
Now, given those four marginal counts and $H_0: F_0\left(\cdot\right) = F_1\left(\cdot\right)$, one can see that $d_{1j}$ determines the whole table and actually $d_{1j}$ follows a hypergeometric distribution
\[
  P\left(D_{1j} = d\right) =
  \frac{
    C_{Y_0\left(\tau_j\right)}^{d_{0j}}
    C_{Y_1\left(\tau_j\right)}^{d_j - d_{0j}}
  }{
    C_{Y\left(\tau_j\right)}^{d_j}
  }
  ,
\]
where $d$ ranges such that
\[
  \begin{aligned}
    d &\geq 0    \\
    d_j - d &\geq 0    \\
    Y_1\left(\tau_j\right) - d &\geq 0    \\
    Y_0\left(\tau_j\right) - \left(d_j - d\right) &\geq 0
  \end{aligned}
\]
Therefore
\[
  \mathrm{max}\left(0, d_j - Y_0\left(\tau_j\right)\right)
  \leq d
  \leq \mathrm{min}\left(d_j, Y_1\left(\tau_j\right)\right)
  .
\]
And it's easy to know that
\[
  \begin{aligned}
    \mathrm{E}_j &= \mathrm{E}\left(D_{1j}\right) &&= \frac{Y_1\left(\tau_j\right)d_j}{Y\left(\tau_j\right)}    \\
    \mathrm{V}_j &= \mathrm{Var}\left(D_{1j}\right) &&=
                                       \frac{Y\left(\tau_j\right) - Y_1\left(\tau_j\right)}{Y\left(\tau_j\right) - 1}
                                       \cdot Y_1\left(\tau_j\right)
                                       \left(
                                       \frac{d_j}{Y\left(\tau_j\right)}
                                       \right)
                                       \left(
                                       1 - \frac{d_j}{Y\left(\tau_j\right)}
                                       \right)    \\
    && &= \frac{
       Y_0\left(\tau_j\right)Y_1\left(\tau_j\right)d_j\left(Y\left(\tau_j\right) - d_j\right)
       }{
       Y\left(\tau_j\right)^2
       \left(Y\left(\tau_j\right) - 1\right)
       }
  \end{aligned}
\]
And denote the observation $O_j = d_{1j}$. And we can define for over the whole time points
\[
  \begin{aligned}
    & \mathrm{O} = \sum\limits_{j = 1}^k O_j    \\
    & \mathrm{E} = \sum\limits_{j = 1}^k \mathrm{E}_j    \\
    & \mathrm{V} = \sum\limits_{j = 1}^k \mathrm{V}_j
  \end{aligned}
\]
And the test statistic is argued to follow under $H_0$:
\[
  Z = \frac{O - \mathrm{E}}{\sqrt{\mathrm{V}}}
  \overset{apx}{\sim}\; N\left(0, 1\right)
\]


\bibliographystyle{plainnat}
\bibliography{../ref}





\end{document}


%%% Local Variables:
%%% mode: latex
%%% TeX-master: t
%%% End:
