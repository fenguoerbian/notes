\documentclass[a4paper,12pt]{article}
\usepackage{geometry}
\geometry{left=2.5cm,right=2.5cm,top=2.5cm,bottom=2.5cm}
\renewcommand{\textfraction}{0.15}
\renewcommand{\topfraction}{0.85}
\renewcommand{\bottomfraction}{0.65}
\renewcommand{\floatpagefraction}{0.60}
\usepackage{amsmath}
\usepackage{amsfonts}
\usepackage{mathrsfs}
\usepackage{amsthm}
\usepackage{extarrows}
\usepackage{bm}
% \newcommand{\bm}{\symbfit}    % `bm` confilicts with `unicode-math`. In that case use \symbfit for bold math symbols
\usepackage{graphicx}
\usepackage[section]{placeins}
\usepackage{flafter}
\usepackage{array}
\usepackage{caption}
\usepackage{subcaption}
\usepackage{color}
\usepackage{multirow}
\usepackage{natbib}
% \usepackage{enumerate}
\usepackage{enumitem}    % more flexible than `enumerate` package, the reference will carry the whole label appearance, not just the counter, unlike the `enumerate` package.
\usepackage{upgreek}    % 'upgreek' letters

% \pdfstringdefDisableCommands{\let\bm=\relax}

\newtheorem{thm}{Theorem}
\newtheorem{cor}{Corollary}
\newtheorem{assum}{Assumption}
\newtheorem{rem}{Remark}
\newtheorem{lem}{Lemma}
\newtheorem{prop}{Proposition}

\setcounter{topnumber}{5}    % Maximum number of floats that can appear at the top of a text page; default 2. 
\setcounter{bottomnumber}{5}   % Maximum number of floats that can appear at the bottom of a text page; default 1. 
\setcounter{totalnumber}{10}    % Maximum number of floats that can appear on a text page; default 3. 

\DeclareMathOperator*{\argmaxdown}{arg\,max}
\DeclareMathOperator*{\argmindown}{arg\,min}
\DeclareMathOperator{\argmax}{arg\,max}
\DeclareMathOperator{\argmin}{arg\,min}

% cross-reference to other files
% the externaldocument should be compiled 
% (and at least twice if you're using xr-hyper)
% \usepackage{xr-hyper}
% \usepackage{xr}

% --- external document (ordinary setting) ---
% \externaldocument{external_tex_file}
% --- end of ordinary setting ---

% --- external document (overleaf setting) ---
% externaldocument settings for Overleaf
% \makeatletter
% \newcommand*{\addFileDependency}[1]{% argument=file name and extension
% \typeout{(#1)}
% \@addtofilelist{#1}
% \IfFileExists{#1}{}{\typeout{No file #1.}}
% }
%   \makeatother
%   \newcommand*{\myexternaldocument}[1]{%
%   \externaldocument{#1}%
%   \addFileDependency{#1.tex}%
%   \addFileDependency{#1.aux}%
% }
%   \myexternaldocument{external_tex_file}
%   --- end of overleaf setting ---

%   mathtools can be used to define labeling format for equations
%   one can use \eqref for a reference to a labeled equation.
\usepackage{mathtools}
\newtagform{supp}{(S-}{)}    % define a equation labeling format for suppliment
\usetagform{supp}            % use the supp format
\usetagform{default}         % use the default format

\usepackage{algorithmic}
\usepackage{algorithm}
\renewcommand{\algorithmicrequire}{\textbf{Input:}}
\renewcommand{\algorithmicensure}{\textbf{Output:}}

% package for hyperlinks
% It's error-prone because hyper link is quite difficult
% due to the fact the typesetting environment is complex.
% So you can disable this package and finish the document.
% Then sort out the hyperlink thing.
\usepackage[colorlinks,linkcolor=red,anchorcolor=blue,citecolor=green,CJKbookmarks=True]{hyperref}

% package for displaying highlighted codes
\usepackage{minted}

% package for input codec and output rendering font
\usepackage[utf8]{inputenc}    % it is always good practice to use utf8
% you can also try latin1, latin2, cp1252 and cp1250
\usepackage[T1]{fontenc}    % the default is T0, which only contains 128 characters
% you can try T1, T2A, T2B
% you can refer to https://www.overleaf.com/learn/latex/international_language_support#Font_encoding
% for more details.


\title{Stratified v.s. Unstratified Analysis}
\author{Chao Cheng}
\date{\today}



\begin{document}
\maketitle
\tableofcontents{}

\section{Introduction}
\label{sec:introduction}

In this note we will talk about the Cox's proportional hazards (Cox's PH) model. And more specifically, what will happen when unstratified analysis is used for a data where stratified analysis is the true model. 

\section{A simple parametric model}
\label{sec:simple-param-model}

Consider the Weibull distribution, denote $T \sim W(p, \lambda)$. Then

\[
  \begin{aligned}
    & f\left(t\right) = p\lambda^pt^{p - 1}\mathrm{exp}\left(-\left(\lambda t\right)^p\right)    \\
    & F\left(t\right) = 1 - \mathrm{exp}\left(-\left(\lambda t\right)^p\right)
      \quad\quad S\left(t\right) = \mathrm{exp}\left(-\left(\lambda t\right)^p\right)    \\
    & h\left(t\right) = p\lambda^pt^{p - 1}    \\
    & H\left(t\right) = \left(\lambda t\right)^p    \\
    & \mathrm{E}\left(T\right) = \frac{1}{\lambda} \cdot \Gamma\left(1 + \frac{1}{p}\right)
      \quad\quad \mathrm{Var}\left(T\right) = \frac{1}{\lambda^2}\left(
      \Gamma\left(1 + \frac{2}{p}\right)
      - \Gamma\left(1 + \frac{1}{p}\right)
      \right)    \\
    & \mathrm{E}\left(T^m\right) = \frac{1}{\lambda^m}\Gamma\left(1 + \frac{m}{p}\right)
  \end{aligned}
\]

Then the likelihood is

\begin{equation}
  \label{eq:weibull_likelihood}
  \begin{aligned}
    L\left(t_1, \cdots, t_n\middle|p, \lambda_1, \cdots, \lambda_n\right)
    &= \prod\limits_{i = 1}^n\left(
    f(t_i)^{\delta_i}\left(1 - F\left(t_i\right)\right)^{1 - \delta_i}
    \right)
    = \prod\limits_{i = 1}^n\left(
    h\left(t_i\right)^{\delta_i}S\left(t_i\right)
    \right)    \\
    &= \prod\limits_{i = 1}^n
      \left(p\lambda_i^{p}t_i^{p - 1}\right)^{\delta_i}
      \mathrm{exp}\left(-\left(\lambda_i t_i\right)^p\right)
  \end{aligned}
\end{equation}

where $\delta_i = 1$ means an event is observed for $i$. Otherwise $\delta_i = 0$ represents censor is observed. Note that in \eqref{eq:weibull_likelihood}, we assume all subjects share the same $p$ in the Weibull distribution, but their $\lambda$s can be different. 



\section{Cox model}
\label{sec:cox-model}

For a Cox model, the key assumption is constant hazard ratio, that is
\[
  h\left(t\middle|Z\right) = h_0\left(t\right)\cdot e^{\beta Z}
\]

And if we plug-in the Weibull distribution, that is $h_0\left(t\right) = p\lambda_0^pt^{p - 1}$. Then
\[
  h\left(t\middle|Z\right)
  = h_0\left(t\right)\cdot e^{\beta Z}
  = p\lambda_0^pt^{p - 1} \cdot e^{\beta Z}
  = p \left(\lambda_0 e^{\beta Z / p}\right)^p t^{p - 1}
\]

Here for our purpose, we let $Z_i \in \left\{0, 1\right\}$ denote the treatment(1) or control(0) group. And in this case, the data likelihood \eqref{eq:weibull_likelihood} becomes
\[
  \begin{aligned}
    L\left(t_1, \cdots, t_n\middle| p, \lambda, \beta\right)
    &= \prod\limits_{i = 1}^n
    \left(p \left(\lambda e^{\beta Z_i/p}\right)^pt_i^{p - 1}\right)^{\delta_i}
    \mathrm{exp}\left(-\left(\lambda e^{\beta Z_i/p} t_i\right)^p\right)    \\
    &= \prod\limits_{i = 1}^n
    \left(p \lambda^p e^{\beta Z_i} t_i^{p - 1}\right)^{\delta_i}
    \mathrm{exp}\left(-\left(\lambda t_i\right)^p e^{\beta Z_i}\right)
  \end{aligned}
\]

And the loglikelihood is
\begin{equation}
  \label{eq:weibull_loglikelihood}
  \begin{aligned}
    \mathrm{log}L
    &= \sum\limits_{i = 1}^n
      \delta_i\left(
      \mathrm{log}p + p\mathrm{log}\lambda + \beta Z_i + \left(p - 1\right)t_i
      \right)
      - \left(\lambda t_i\right)^p e^{\beta Z_i}    \\
    &= n_{evt}\left(\mathrm{log}p + p \mathrm{log}\lambda\right)
      + \sum\limits_{i = 1}^n\delta_i\left(\beta Z_i + \left(p - 1\right)t_i\right)
      - \lambda^p\sum\limits_{i = 1}^nt_i^pe^{\beta Z_i}
  \end{aligned}
\end{equation}

Use the profile likelihood method, first we fix $\beta$ and $p$ to maximize $\mathrm{log}L$ w.r.t $\lambda$:
\[
  \begin{aligned}
    \frac{\partial\mathrm{log}L}{\partial \lambda}
    = \frac{n_{evt}p}{\lambda}
    - p\lambda^{p - 1}\sum\limits_{i = 1}^nt_i^pe^{\beta Z_i}
  \end{aligned}
\]
Set this to 0 we have
\[
  \hat{\lambda} = \left(
    \frac{n_{evt}}{\sum\limits_{i = 1}^nt_i^pe^{\beta Z_i}}
  \right)^{1 / p}
\]
Plug this back into \eqref{eq:weibull_loglikelihood} will give us
\[
  \begin{aligned}
    \mathrm{log}L
    &= n_{evt}\left(
      \mathrm{log}p + \mathrm{log}n_{evt}
      - \mathrm{log}\left(\sum\limits_{i = 1}^nt_i^pe^{\beta Z_i}\right)
      \right)
      + \sum\limits_{i = 1}^n\delta_i\left(\beta Z_i + \left(p - 1\right)t_i\right)
      - \frac{n_{evt}}{\sum\limits_{i = 1}^nt_i^pe^{\beta Z_i}} \cdot \sum\limits_{i = 1}^nt_i^pe^{\beta Z_i}    \\
    &= n_{evt}\left(
      \mathrm{log}p + \mathrm{log}n_{evt}
      - \mathrm{log}\left(\sum\limits_{i = 1}^nt_i^pe^{\beta Z_i}\right)
      \right)
      + \sum\limits_{i = 1}^n\delta_i\left(\beta Z_i + \left(p - 1\right)t_i\right)
      - n_{evt}
  \end{aligned} 
\]
Unfortunately, there's no analytical solution to $p$ even when we fixed $\beta$. So let's consider a simpler case where we fix $p = 1$, i.e. Exponential distribution. Then this loglikelihood becomes
\[
  \begin{aligned}
    \mathrm{log}L
    &= n_{evt}\left(
      \mathrm{log}n_{evt}
      - \mathrm{log}\left(\sum\limits_{i = 1}^nt_ie^{\beta Z_i}\right)
      \right)
      + \sum\limits_{i = 1}^n\delta_i\beta Z_i
      - n_{evt}    \\
    &\overset{w.r.t\; \beta}{\propto}
      -n_{evt}\mathrm{log}\left(\sum\limits_{i = 1}^nt_ie^{\beta Z_i}\right)
      +  \sum\limits_{i = 1}^n\delta_i\beta Z_i    \\
    &= \sum\limits_{i = 1}^n\delta_i\left(
      \beta Z_i - \mathrm{log}\left(\sum\limits_{i = 1}^nt_ie^{\beta Z_i}\right)
      \right)    \\
    &= \sum\limits_{i = 1}^n\delta_i\mathrm{log}
      \frac{e^{\beta Z_i}}{\sum\limits_{i = 1}^nt_ie^{\beta Z_i}}
  \end{aligned}
\]
Therefore to maximize $\mathrm{log}L$ with respect to $\beta$, is equivalent to maximize the following term
\begin{equation}
  \label{eq:beta_mle_objective}
  \prod\limits_{i = 1}\left(
    \frac{e^{\beta Z_i}}{\sum\limits_{i = 1}^nt_ie^{\beta Z_i}}
  \right)^{\delta_i}
\end{equation}

\textbf{Note:} \eqref{eq:beta_mle_objective} can be seen as objective function for $\beta$'s MLE and it is \textbf{different} from the partial likelihood used in Cox regression.



\subsection{Stratified setting}
\label{sec:stratified-setting}

Now let's consider the stratified setting, with $K$ strata. Then for each stratum $k\in\left\{1, \cdots, K\right\}$, the Weibull distribution for control group is $W\left(p_k, \lambda_k\right)$, which means the hazard is
\[
  h_{0, k}\left(t\right) = p_k\lambda_k^{p_k}t^{p_k - 1}.
\]

Assume the constant hazard ratio is $e^\beta$. Then the hazard for treatment group is $h_{1, k}\left(t\right) = h_{0, k}\left(t\right)e^\beta$. Therefore
\[
  h_k\left(t\middle|Z\right) = h_{0, k}\left(t\right)e^{\beta Z},\quad Z\in\left\{0, 1\right\}. 
\]




\bibliographystyle{plainnat}
\bibliography{../../ref}





\end{document}


%%% Local Variables:
%%% mode: latex
%%% TeX-master: t
%%% End:
