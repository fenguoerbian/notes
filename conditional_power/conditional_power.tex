\documentclass[a4paper,12pt]{article}
% \usepackage{ctex}
\usepackage{geometry}
\geometry{left=2.5cm,right=2.5cm,top=2.5cm,bottom=2.5cm}
\renewcommand{\textfraction}{0.15}
\renewcommand{\topfraction}{0.85}
\renewcommand{\bottomfraction}{0.65}
\renewcommand{\floatpagefraction}{0.60}
\usepackage{amsmath}
\usepackage{amsfonts}
\usepackage{mathrsfs}
\usepackage{amsthm}
\usepackage{extarrows}
\usepackage{bm}
% \newcommand{\bm}{\symbfit}    % `bm` confilicts with `unicode-math`. In that case use \symbfit for bold math symbols
\usepackage{graphicx}
\usepackage[section]{placeins}
\usepackage{flafter}
\usepackage{array}
\usepackage{caption}
\usepackage{subcaption}
\usepackage{color}
\usepackage{multirow}
\usepackage{natbib}
% \usepackage{enumerate}
\usepackage{enumitem}    % more flexible than `enumerate` package, the reference will carry the whole label appearance, not just the counter, unlike the `enumerate` package.
\usepackage{upgreek}    % 'upgreek' letters

% \pdfstringdefDisableCommands{\let\bm=\relax}

\newtheorem{thm}{Theorem}
\newtheorem{cor}{Corollary}
\newtheorem{assum}{Assumption}
\newtheorem{rem}{Remark}
\newtheorem{lem}{Lemma}

\setcounter{topnumber}{5}    % Maximum number of floats that can appear at the top of a text page; default 2. 
\setcounter{bottomnumber}{5}   % Maximum number of floats that can appear at the bottom of a text page; default 1. 
\setcounter{totalnumber}{10}    % Maximum number of floats that can appear on a text page; default 3. 

\DeclareMathOperator*{\argmaxdown}{arg\,max}
\DeclareMathOperator*{\argmindown}{arg\,min}
\DeclareMathOperator{\argmax}{arg\,max}
\DeclareMathOperator{\argmin}{arg\,min}

% cross-reference to other files
% the externaldocument should be compiled 
%   (and at least twice if you're using xr-hyper)
% \usepackage{xr-hyper}
% \usepackage{xr}

% --- external document (ordinary setting) ---
% \externaldocument{external_tex_file}
% --- end of ordinary setting ---

% --- external document (overleaf setting) ---
% externaldocument settings for Overleaf
% \makeatletter
% \newcommand*{\addFileDependency}[1]{% argument=file name and extension
%   \typeout{(#1)}
%   \@addtofilelist{#1}
%   \IfFileExists{#1}{}{\typeout{No file #1.}}
% }
% \makeatother
% \newcommand*{\myexternaldocument}[1]{%
%     \externaldocument{#1}%
%     \addFileDependency{#1.tex}%
%     \addFileDependency{#1.aux}%
% }
% \myexternaldocument{external_tex_file}
% --- end of overleaf setting ---

% mathtools can be used to define labeling format for equations
% one can use \eqref for a reference to a labeled equation.
\usepackage{mathtools}
\newtagform{supp}{(S-}{)}    % define a equation labeling format for suppliment
\usetagform{supp}            % use the supp format
\usetagform{default}         % use the default format

\usepackage{algorithmic}
\usepackage{algorithm}
\renewcommand{\algorithmicrequire}{\textbf{Input:}}
\renewcommand{\algorithmicensure}{\textbf{Output:}}

% package for hyperlinks
% It's error-prone because hyper link is quite difficult
% due to the fact the typesetting environment is complex.
% So you can disable this package and finish the document.
% Then sort out the hyperlink thing.
\usepackage[colorlinks,linkcolor=red,anchorcolor=blue,citecolor=green,unicode=true]{hyperref}

% package for displaying highlighted codes
\usepackage{minted}

% package for input codec and output rendering font
\usepackage[utf8]{inputenc}    % it is always good practice to use utf8
                               % you can also try latin1, latin2, cp1252 and cp1250
\usepackage[T1]{fontenc}    % the default is T0, which only contains 128 characters
                            % you can try T1, T2A, T2B
% you can refer to https://www.overleaf.com/learn/latex/international_language_support#Font_encoding
% for more details.

\usepackage{soul}    % \ul{} for underline, \st{} for strikethrough/overstriking, \hl{} for highlingt


\title{Conditional Power}
\author{Chao Cheng}
\date{\today}

\begin{document}
\maketitle
\tableofcontents{}

\section{Introduction}
\label{sec:introduction}

The conditional power is defined as the probability to reject the null hypothersis at \textbf{Final} analysis given the current collected data.

\section{Mean Test}
\label{sec:mean-test}

\subsection{Normal distribution, variance known}
\label{sec:norm-distr-vari}

\subsubsection{One-sample mean test}
\label{sec:one-sample-mean}

Let the sample $x_1, \cdots, x_n \sim N\left(\mu, \sigma^2\right)$, where $\sigma^2$ is known. To simplify the discussion here, we assume an one-sided test
\[
  H_0:\; \mu \leq \mu_0,\quad H_1:\; \mu > \mu_0
\]
The reject rule is rejecting $H_0$ if $\bar{x} > c$ for some $c$. Since $\bar{x}\sim N\left(\mu, \sigma^2 / n\right)$, $c$ can be determined by type-I error $\alpha$:
\[
  P\left(\bar{x} > c\middle| H_0\right) \leq \alpha
  .
\]
By Neyman-Pearson Lemma, and stochastical monotone property of normal distribution, to fully utilize the type-I error, it is
\[
  P\left(\bar{x} > c \middle| \mu = \mu_0\right)
  = P\left(\frac{\bar{x} - \mu_0}{\sqrt{\sigma^2 / n}}
    > \frac{c - \mu_0}{\sqrt{\sigma^2 / n}}\right)
  = P\left(Z > \frac{c - \mu_0}{\sqrt{\sigma^2 / n}}\right)
  = \alpha
  .
\]
Hence $c = \mu_0 - z_{\alpha}\sqrt{\sigma^2 / n}$, where $z_\alpha = \Phi^{-1}\left(\alpha\right)$ is the (left-)$\alpha$ quantile of normal distribution. For more details about how this test is constructed, one can refer to the \textbf{ttest.pdf} notes in this repo.
\par
The probability to reject the null hypothersis at some given $\mu_1$ can be computed as
\[
  \begin{aligned}
    & P\left(\bar{x} > c\middle| \mu = \mu_1\right)    \\
    =& P\left(\frac{\bar{x} - \mu_1}{\sqrt{\sigma^2 / n}}
      > \frac{c - \mu_1}{\sqrt{\sigma^2 / n}}\right)    \\
    =& P\left(
      Z > -z_{\alpha} + \frac{\mu_0 - \mu_1}{\sqrt{\sigma^2 / n}}\right)    \\
    =& 1 - \Phi\left(-z_{\alpha} + \frac{\mu_0 - \mu_1}{\sqrt{\sigma^2 / n}}\right) = 1 - \beta
    ,
  \end{aligned}
\]
where $\beta$ is the type-II error. Then the samplesize is determined by
\[
  n = \frac{\sigma^2\left(z_{\alpha} + z_{\beta}\right)^2}{\left(\mu_0 - \mu_1\right)^2}.
\]
\par
Now, assume we have collected $n_1$ samples, which means there are $n_2 = n - n_1$ more samples to come. And the current statistics is $\bar{x}_1 = \frac{1}{n_1}\sum\limits_{i = 1}^{n_1}x_i$ and the final statistics is $\bar{x} = \frac{1}{n_1 + n_2}\sum\limits_{i = 1}^{n_1 + n_2}x_i$. The statistics from the second part is $\bar{x}_2 = \frac{1}{n_2}\sum\limits_{i = n_1 + 1}^{n_1 + n_2}x_i$. Note that these ($\bar{x}_1$, $\bar{x}_2$ and $\bar{x}$) are all sufficient statistics for $\mu$ (based on part1, part1 and all parts).
\par
Here we know
\[
  \bar{x}_1 \sim N\left(\mu, \sigma^2 / n_1\right)
  ,\quad
  \bar{x}_2 \sim N\left(\mu, \sigma^2 / n_2\right)
  ,\quad
  \bar{x}_1 \perp \bar{x}_2
  .
\]
And $\bar{x} = \frac{n_1}{n_1 + n_2}\bar{x}_1 + \frac{n_2}{n_1 + n_2}\bar{x}_2$. So the conditional power, which is the probability to reject the hypothesis at final analysis given the current collected data is
\begin{equation}
  \label{eq:cp_normal_sigma_known_one_sample}
  \begin{aligned}
    & P\left(\bar{x} > c \middle| \bar{x}_1\right)    \\
    =& P\left(
      \frac{n_1}{n_1 + n_2}\bar{x}_1 + \frac{n_2}{n_1 + n_2}\bar{x}_2 > c\middle| \bar{x}_1
    \right)    \\
    =& P\left(
      \bar{x}_2 > \frac{\left(n_1 + n_2\right)c - n_1\bar{x}_1}{n_2}
    \right)    \\
    =& P\left(
      \frac{\bar{x}_2 - \mu}{\sqrt{\sigma^2/n_2}}
      > \frac{
        \frac{\left(n_1 + n_2\right)c - n_1\bar{x}_1}{n_2}- \mu
      }{\sqrt{\sigma^2/n_2}}
    \right)    \\
    =& P\left(
      Z > \frac{
        -z_{\alpha}\sqrt{\left(n_1 + n_2\right)\sigma^2}
        + n_1\left(\mu_0 - \bar{x}_1\right)
        + n_2\left(\mu_0 - \mu\right)
      }{\sqrt{n_2\sigma^2}}
    \right)    \\
    =& 1 - \Phi\left(
      \frac{
        -z_{\alpha}\sqrt{\left(n_1 + n_2\right)\sigma^2}
        + n_1\left(\mu_0 - \bar{x}_1\right)
        + n_2\left(\mu_0 - \mu\right)
      }{\sqrt{n_2\sigma^2}}
    \right)
  \end{aligned}
\end{equation}
\textbf{Note:} in previous discussions there are some abuse of the notation. $\bar{x}_1$ and $\bar{x}_2$ can both refer to the random variable, and there realization. But I hope the context is clear and not much confusion.

\subsubsection{Two-sample mean test}
\label{sec:two-sample-mean}



\section{Rate Test}
\label{sec:rate-test}

\section{Log Rank Test}
\label{sec:log-rank-test}

\section{Sample Size Re-estimation/Re-calculation}
\label{sec:sample-size-re}








\clearpage
\appendix


\end{document}


%%% Local Variables:
%%% mode: latex
%%% TeX-master: t
%%% End:
