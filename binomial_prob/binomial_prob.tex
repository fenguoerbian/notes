\documentclass[a4paper,12pt]{article}
\usepackage{geometry}
\geometry{left=2.5cm,right=2.5cm,top=2.5cm,bottom=2.5cm}
\renewcommand{\textfraction}{0.15}
\renewcommand{\topfraction}{0.85}
\renewcommand{\bottomfraction}{0.65}
\renewcommand{\floatpagefraction}{0.60}
\usepackage{amsmath}
\usepackage{amsfonts}
\usepackage{mathrsfs}
\usepackage{amsthm}
\usepackage{extarrows}
\usepackage{bm}
% \newcommand{\bm}{\symbfit}    % `bm` confilicts with `unicode-math`. In that case use \symbfit for bold math symbols
\usepackage{graphicx}
\usepackage[section]{placeins}
\usepackage{flafter}
\usepackage{array}
\usepackage{caption}
\usepackage{subcaption}
\usepackage{color}
\usepackage{multirow}
\usepackage{natbib}
% \usepackage{enumerate}
\usepackage{enumitem}    % more flexible than `enumerate` package, the reference will carry the whole label appearance, not just the counter, unlike the `enumerate` package.
\usepackage{upgreek}    % 'upgreek' letters

% \pdfstringdefDisableCommands{\let\bm=\relax}

\newtheorem{thm}{Theorem}
\newtheorem{cor}{Corollary}
\newtheorem{assum}{Assumption}
\newtheorem{rem}{Remark}
\newtheorem{lem}{Lemma}

\setcounter{topnumber}{5}    % Maximum number of floats that can appear at the top of a text page; default 2. 
\setcounter{bottomnumber}{5}   % Maximum number of floats that can appear at the bottom of a text page; default 1. 
\setcounter{totalnumber}{10}    % Maximum number of floats that can appear on a text page; default 3. 

\DeclareMathOperator*{\argmaxdown}{arg\,max}
\DeclareMathOperator*{\argmindown}{arg\,min}
\DeclareMathOperator{\argmax}{arg\,max}
\DeclareMathOperator{\argmin}{arg\,min}

% cross-reference to other files
% the externaldocument should be compiled 
%   (and at least twice if you're using xr-hyper)
% \usepackage{xr-hyper}
% \usepackage{xr}

% --- external document (ordinary setting) ---
% \externaldocument{external_tex_file}
% --- end of ordinary setting ---

% --- external document (overleaf setting) ---
% externaldocument settings for Overleaf
% \makeatletter
% \newcommand*{\addFileDependency}[1]{% argument=file name and extension
%   \typeout{(#1)}
%   \@addtofilelist{#1}
%   \IfFileExists{#1}{}{\typeout{No file #1.}}
% }
% \makeatother
% \newcommand*{\myexternaldocument}[1]{%
%     \externaldocument{#1}%
%     \addFileDependency{#1.tex}%
%     \addFileDependency{#1.aux}%
% }
% \myexternaldocument{external_tex_file}
% --- end of overleaf setting ---

% mathtools can be used to define labeling format for equations
% one can use \eqref for a reference to a labeled equation.
\usepackage{mathtools}
\newtagform{supp}{(S-}{)}    % define a equation labeling format for suppliment
\usetagform{supp}            % use the supp format
\usetagform{default}         % use the default format

\usepackage{algorithmic}
\usepackage{algorithm}
\renewcommand{\algorithmicrequire}{\textbf{Input:}}
\renewcommand{\algorithmicensure}{\textbf{Output:}}

% package for hyperlinks
% It's error-prone because hyper link is quite difficult
% due to the fact the typesetting environment is complex.
% So you can disable this package and finish the document.
% Then sort out the hyperlink thing.
\usepackage[colorlinks,linkcolor=red,anchorcolor=blue,citecolor=green,CJKbookmarks=True]{hyperref}

% package for displaying highlighted codes
\usepackage{minted}

% package for input codec and output rendering font
\usepackage[utf8]{inputenc}    % it is always good practice to use utf8
                               % you can also try latin1, latin2, cp1252 and cp1250
\usepackage[T1]{fontenc}    % the default is T0, which only contains 128 characters
                            % you can try T1, T2A, T2B
% you can refer to https://www.overleaf.com/learn/latex/international_language_support#Font_encoding
% for more details.


\title{Test for the probability of a binomial distribution}
\author{Chao Cheng}
\date{\today}



\begin{document}
\maketitle

For an i.i.d sample from a bernoulli distribution
\[
  x_1, \cdots, x_n \overset{\mathrm{i.i.d.}}{\sim} Bernoulli(p)
  ,
\]
The likelihood of the data is
\[
  f\left(x_1, \cdots, x_n\right)
  = \prod\limits_{i = 1}^np^{x_i}\left(1 - p\right)^{1 - x_i}
  = p^{\sum x_i}\left(1 - p\right)^{n - \sum x_i}
  .
\]
MLE for $p$ is $\bar{x} = \frac{1}{n}\sum x_i$ and
\[
  \sum\limits_{i = 1}^nx_i \sim Binom(n, p)
  .
\]

So here are mainly two situations: One is to test the probability $p$ against some given value $p_0$. The other is to compare the probability between two independent random samples $x_1, \cdots, x_n$ and $y_1, \cdots, y_m$.
\begin{enumerate}[label = Case\arabic*:]
\item One sample $x_1, \cdots, x_n$ from $Bernoulli(p)$, and test $p$ against a given $p_0$.
\item Two samples: $x_1, \cdots, x_2$ from $Bernoulli(p_1)$ and $y_1, \cdots, y_m$ from $Bernoulli(p_2)$. And test whether $p_1 = p_2$.
\end{enumerate}

\section{Normal approximation}
\label{sec:normal-approximation}

\subsection{Case 1}
\label{sec:case-1}

Note that
\[
  \mathrm{E}X = p,\quad\mathrm{Var}X = p\left(1 - p\right).
\]
Then by CLT we have
\[
  \bar{x} \overset{\mathrm{asymp}}{\sim} N\left(p,\;\frac{p\left(1 - p\right)}{n}\right)
  .
\]
For $H_0:\;p = p_0$, we propose a test statistic
\[
  Z = \frac{\bar{x} - p_0}{\sqrt{\frac{p_0\left(1 - p_0\right)}{n}}}
  .
\]
Then $Z$ is asymptotically standard normal under $H_0$.
\par
Also we know that under $H_1$:
\[
  \begin{aligned}
    Z =& \frac{\bar{x} - p_0}{\sqrt{\frac{p_0\left(1 - p_0\right)}{n}}}    \\
    =& \frac{\bar{x} - p}{\sqrt{\frac{p\left(1 - p\right)}{n}}}
    \cdot \sqrt{\frac{p\left(1 - p\right)}{p_0\left(1 - p_0\right)}}
    + \frac{p - p_0}{\sqrt{\frac{p_0\left(1 - p_0\right)}{n}}}    \\
    \sim& N\left(
      \frac{p - p_0}{\sqrt{\frac{p_0\left(1 - p_0\right)}{n}}}
      ,\quad
      \frac{p\left(1 - p\right)}{p_0\left(1 - p_0\right)}
    \right)
    .
  \end{aligned}
\]
So the power of the test can be easily computed.

\subsection{Case 2}
\label{sec:case-2}

So we have
\[
  \bar{x} \overset{\mathrm{asymp}}{\sim} N\left(p_1,\;\frac{p_1\left(1 - p_1\right)}{n}\right)
  ,\quad\text{and}\quad
  \bar{y} \overset{\mathrm{asymp}}{\sim} N\left(p_2,\;\frac{p_2\left(1 - p_2\right)}{m}\right)
  .
\]
A test statistic can be
\[
  Z = \frac{\bar{x} - \bar{y}}{
    \sqrt{
      \hat{p}\left(1 - \hat{p}\right)
      \left(\frac{1}{n} + \frac{1}{m}\right)
    }
  }
  ,
\]
where $\hat{p} = \frac{n\bar{x} + m\bar{y}}{n + m}$. This test statistic can be found at
\begin{verbatim}
https://stats.stackexchange.com/questions/361015/
  proof-of-the-standard-error-of-the-distribution-between-two-normal-distributions/
  361048#361048

https://stats.stackexchange.com/questions/113602/
  test-if-two-binomial-distributions-are-statistically-different-from-each-other
\end{verbatim}
Here this $\hat{p}\left(1 - \hat{p}\right)$ can be seen as an estimate for the variance $p\left(1 - p\right)$ when $H_0$ is true by directly plugging in $\hat{p}$. This is \textbf{NOT} a pooled variance for these two samples, which should always be no greater than $\hat{p}\left(1 - \hat{p}\right)$.
\par
The power of this test statistic is hard to compute under $H_1$.
\par
\textbf{Note: } One can also use the same idea in the ``t-test.pdf'' notes and propose the test statistic
\[
  T = \frac{\bar{x} - p_0}{\sqrt{S_x / n}}
  ,
\]
where $S_x = \frac{1}{n - 1}\sum\limits_{i = 1}^n\left(x_i - \bar{x}\right)$ for Case 1.
\par
And for Case 2
\[
  T = \frac{\bar{x} - \bar{y} - \Delta}{\sqrt{
      \left(\frac{1}{n} + \frac{1}{m}\right)S_p
    }}
  ,
\]
where $S_p^2 = \frac{\left(n - 1\right)S_x^2 + \left(m - 1\right)S_y^2}{n + m - 2}$ and $\Delta = p_1 - p_2$. But again it is hard to evaluate the testing power of these statistics.


\section{Chi-square approximation}
\label{sec:chi-square-appr}

See the notes of ``chisq\_test.pdf'' for details.


\section{Exact test}
\label{sec:exact-test}

\subsection{Case 1: Clopper-Pearson test}
\label{sec:case-1:-clopper}

The Clopper-Pearson method is an early method. It's called exact method because it's directly based on p.m.f of binomial distribution. Let $X = \sum\limits_{i = 1}^nx_i$. Then $X\sim Binom(n, p)$ and the p.m.f is
\[
  P\left(X = x\middle|p\right) = C_n^xp^x\left(1 - p\right)^{n - x}
\]
for $x = 0, 1, \cdots, n$. So let's recall that p-value is the probability under $H_0$ that something as or more extreme than what we have observed happens. Then after observing $X = x_0$, for one-sided test:
\begin{itemize}
\item $H_0: p \leq p_0$ against $H_1: p > p_0$ for some given $p_0$. The p-value is
  \[
    p_{val} = \sum\limits_{x = x_0}^nP\left(X = x\middle|p_0\right)
    .
  \]
  
\item $H_0: p \geq p_0$ against $H_1: p < p_0$ for some given $p_0$. The p-value is
  \[
    p_{val} = \sum\limits_{x = 0}^{x_0}P\left(X = x\middle|p_0\right)
    .
  \]
\end{itemize}
For the two-sided test. This is a little complicated. Let index set
\[
  \mathcal{I} = \left\{
    x\middle|
    P\left(X = x\middle|p_0\right) \leq P\left(X = x_0\middle|p_0\right)
    ,\quad
    0\leq x \leq n
  \right\}
  .
\]
Then $\mathcal{I}$ contains all possible realizations of $X$ with its probability no greater than the probability of our observation. Then the p-value of $H_0: p = p_0$ is given by
\[
  p_{val} = \sum\limits_{x\in\mathcal{I}}P\left(X = x\middle|p_0\right)
  .
\]


\subsection{Case 2: Fisher's exact test}
\label{sec:case-2:-fishers}





\end{document}



%%% Local Variables:
%%% mode: latex
%%% TeX-master: t
%%% End:
