\documentclass[a4paper,12pt]{article}
\usepackage{geometry}
\geometry{left=2.5cm,right=2.5cm,top=2.5cm,bottom=2.5cm}
\renewcommand{\textfraction}{0.15}
\renewcommand{\topfraction}{0.85}
\renewcommand{\bottomfraction}{0.65}
\renewcommand{\floatpagefraction}{0.60}
\usepackage{amsmath}
\usepackage{amsfonts}
\usepackage{mathrsfs}
\usepackage{amsthm}
\usepackage{extarrows}
\usepackage{bm}
% \newcommand{\bm}{\symbfit}    % `bm` confilicts with `unicode-math`. In that case use \symbfit for bold math symbols
\usepackage{graphicx}
\usepackage[section]{placeins}
\usepackage{flafter}
\usepackage{array}
\usepackage{caption}
\usepackage{subcaption}
\usepackage{color}
\usepackage{multirow}
\usepackage{natbib}
% \usepackage{enumerate}
\usepackage{enumitem}    % more flexible than `enumerate` package, the reference will carry the whole label appearance, not just the counter, unlike the `enumerate` package.
\usepackage{upgreek}    % 'upgreek' letters

% \pdfstringdefDisableCommands{\let\bm=\relax}

\newtheorem{thm}{Theorem}
\newtheorem{cor}{Corollary}
\newtheorem{assum}{Assumption}
\newtheorem{rem}{Remark}
\newtheorem{lem}{Lemma}
\newtheorem{prop}{Proposition}

\setcounter{topnumber}{5}    % Maximum number of floats that can appear at the top of a text page; default 2. 
\setcounter{bottomnumber}{5}   % Maximum number of floats that can appear at the bottom of a text page; default 1. 
\setcounter{totalnumber}{10}    % Maximum number of floats that can appear on a text page; default 3. 

\DeclareMathOperator*{\argmaxdown}{arg\,max}
\DeclareMathOperator*{\argmindown}{arg\,min}
\DeclareMathOperator{\argmax}{arg\,max}
\DeclareMathOperator{\argmin}{arg\,min}

% cross-reference to other files
% the externaldocument should be compiled 
% (and at least twice if you're using xr-hyper)
% \usepackage{xr-hyper}
% \usepackage{xr}

% --- external document (ordinary setting) ---
% \externaldocument{external_tex_file}
% --- end of ordinary setting ---

% --- external document (overleaf setting) ---
% externaldocument settings for Overleaf
% \makeatletter
% \newcommand*{\addFileDependency}[1]{% argument=file name and extension
% \typeout{(#1)}
% \@addtofilelist{#1}
% \IfFileExists{#1}{}{\typeout{No file #1.}}
% }
%   \makeatother
%   \newcommand*{\myexternaldocument}[1]{%
%   \externaldocument{#1}%
%   \addFileDependency{#1.tex}%
%   \addFileDependency{#1.aux}%
% }
%   \myexternaldocument{external_tex_file}
%   --- end of overleaf setting ---

%   mathtools can be used to define labeling format for equations
%   one can use \eqref for a reference to a labeled equation.
\usepackage{mathtools}
\newtagform{supp}{(S-}{)}    % define a equation labeling format for suppliment
\usetagform{supp}            % use the supp format
\usetagform{default}         % use the default format

\usepackage{algorithmic}
\usepackage{algorithm}
\renewcommand{\algorithmicrequire}{\textbf{Input:}}
\renewcommand{\algorithmicensure}{\textbf{Output:}}

% package for hyperlinks
% It's error-prone because hyper link is quite difficult
% due to the fact the typesetting environment is complex.
% So you can disable this package and finish the document.
% Then sort out the hyperlink thing.
\usepackage[colorlinks,linkcolor=red,anchorcolor=blue,citecolor=green,CJKbookmarks=True]{hyperref}

% package for displaying highlighted codes
\usepackage{minted}

% package for input codec and output rendering font
\usepackage[utf8]{inputenc}    % it is always good practice to use utf8
% you can also try latin1, latin2, cp1252 and cp1250
\usepackage[T1]{fontenc}    % the default is T0, which only contains 128 characters
% you can try T1, T2A, T2B
% you can refer to https://www.overleaf.com/learn/latex/international_language_support#Font_encoding
% for more details.


\title{Fisher Information in MLE}
\author{Chao Cheng}
\date{\today}



\begin{document}
\maketitle
\tableofcontents{}


Let i.i.d sample $x_1, \cdots, x_n \sim F_x$. The p.d.f is $f_x\left(x\middle|\theta\right)$ where $\theta$ is the parameter and we use $\theta_0$ to denote the value of underlying parameter $\theta$. Then the log-likelihood
\[
  l = \mathrm{log}L = \sum\limits_{i = 1}^n\mathrm{log}f\left(x_i\middle|\theta\right)
  .
\]

\section{Score function}
\label{sec:score-function}

The {\color{red} score function} is the first derivative ({\color{red} the gradient}) of log-likelihood
\begin{equation}
  \label{eq:score_function}
  S\left(\bm{x}\middle|\theta\right)
  = \frac{\partial l}{\partial \theta}
\end{equation}
The MLE $\hat{\theta}$ makes $S\left(\bm{x}\middle|\theta\right)$ equals to zero:
\[
  S\left(\bm{x}\middle|\hat{\theta}\right) = 0
  .
\]
Also under some regularity conditions
\[
  \mathrm{E}_{\theta_0}\left(
    S\left(\bm{x}\middle|\theta_0\right)
  \right)
  = \bm{\int} \frac{\partial l}{\partial\theta}
  \cdot f\left(\bm{x}\middle|\theta_0\right)
  \mathrm{d}\bm{x}
  = \bm{\int} \sum\limits_{i = 1}^n\frac{\partial\mathrm{log}f\left(x_i\middle|\theta_0\right)}{\partial \theta}
  \prod\limits_{i = 1}^nf\left(x_i\middle|\theta_0\right)
  \mathrm{d}x_1\cdots\mathrm{d}x_n
\]
Note that
\[
  \begin{aligned}
    & \bm{\int} \frac{\partial\mathrm{log}f\left(x_1\middle|\theta_0\right)}{\partial \theta}
      \prod\limits_{i = 1}^nf\left(x_i\middle|\theta_0\right)
      \mathrm{d}x_1\cdots\mathrm{d}x_n    \\
    =& \bm{\int}
       \frac{\partial f\left(x_1\middle|\theta_0\right)}{\partial \theta}
       \cdot \frac{1}{f\left(x_1\middle|\theta_0\right)}
       \prod\limits_{i = 1}^nf\left(x_i\middle|\theta_0\right)
       \mathrm{d}x_1\cdots\mathrm{d}x_n    \\
    =& \bm{\int}
       \left(
       \int \frac{\partial f\left(x_1\middle|\theta_0\right)}{\partial \theta}
       \mathrm{d}x_1
       \right)
  \mathrm{E}_{\theta_0}\left(
    S\left(\bm{x}\middle|\theta_0\right)
  \right)       \prod\limits_{i = 2}^nf\left(x_i\middle|\theta_0\right)
       \mathrm{d}x_2\cdots\mathrm{d}x_n    \\
    =& \frac{\partial}{\partial\theta}
       \int f\left(x_1\middle|\theta_0\right)\mathrm{d}x_1
       \cdot \bm{\int}
       \prod\limits_{i = 2}^nf\left(x_i\middle|\theta_0\right)
       \mathrm{d}x_2\cdots\mathrm{d}x_n    \\
    =& 0
       .
  \end{aligned}
\]
Therefore
\[
  \mathrm{E}_{\theta_0}\left(
    S\left(\bm{x}\middle|\theta_0\right)
  \right)
  = 0
  .
\]

\section{Fisher information}
\label{sec:fisher-information}

Fisher information (matrix) is the second order moment of the score function
\begin{equation}
  \label{eq:def_fisher_info}
  I\left(\theta\right) = \mathrm{E}_{\theta_0}\left(
    S\left(\bm{x}\middle|\theta\right)
    S\left(\bm{x}\middle|\theta\right)^\top
  \right)
  .
\end{equation}
Since $  \mathrm{E}_{\theta_0}\left(S\left(\bm{x}\middle|\theta_0\right)\right)= 0$, the fisher information is the covariance matrix of $S\left(\bm{x}\middle|\theta_0\right)$
\[
  I\left(\theta_0\right) = \mathrm{Var}_{\theta_0}\left(S\left(\bm{x}\middle|\theta_0\right)\right)
  .
\]
Also it can be shown that
\[
  I\left(\theta_0\right) =
  -\mathrm{E}\left(
    \left.\frac{\partial^2l}{\partial\theta^2}\right|_{\theta = \theta_0}
  \right)
  ,
\]
and
\[
  I^\star\left(\theta_0\right) = I\left(\theta_0\right) / n,
\]
where $I^{\star}\left(\theta_0\right)$ is the fisher info for only one single sample. Additionally
\[
  \sqrt{n}\left(
    \hat{\theta} - \theta_0
  \right)
  \overset{D}{\rightarrow}
  N\left(0, I^\star\left(\theta_0\right)^{-1}\right)
  .
\]

\bibliographystyle{plainnat}
\bibliography{../ref}





\end{document}


%%% Local Variables:
%%% mode: latex
%%% TeX-master: t
%%% End:
