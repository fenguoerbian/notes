\documentclass[a4paper,12pt]{article}
\usepackage{geometry}
\geometry{left=2.5cm,right=2.5cm,top=2.5cm,bottom=2.5cm}
\renewcommand{\textfraction}{0.15}
\renewcommand{\topfraction}{0.85}
\renewcommand{\bottomfraction}{0.65}
\renewcommand{\floatpagefraction}{0.60}
\usepackage{amsmath}
\usepackage{amsfonts}
\usepackage{mathrsfs}
\usepackage{amsthm}
\usepackage{extarrows}
\usepackage{bm}
% \newcommand{\bm}{\symbfit}    % `bm` confilicts with `unicode-math`. In that case use \symbfit for bold math symbols
\usepackage{graphicx}
\usepackage[section]{placeins}
\usepackage{flafter}
\usepackage{array}
\usepackage{caption}
\usepackage{subcaption}
\usepackage{color}
\usepackage{multirow}
\usepackage{natbib}
% \usepackage{enumerate}
\usepackage{enumitem}    % more flexible than `enumerate` package, the reference will carry the whole label appearance, not just the counter, unlike the `enumerate` package.
\usepackage{upgreek}    % 'upgreek' letters

% \pdfstringdefDisableCommands{\let\bm=\relax}

\newtheorem{thm}{Theorem}
\newtheorem{cor}{Corollary}
\newtheorem{assum}{Assumption}
\newtheorem{rem}{Remark}
\newtheorem{lem}{Lemma}

\setcounter{topnumber}{5}    % Maximum number of floats that can appear at the top of a text page; default 2. 
\setcounter{bottomnumber}{5}   % Maximum number of floats that can appear at the bottom of a text page; default 1. 
\setcounter{totalnumber}{10}    % Maximum number of floats that can appear on a text page; default 3. 

\DeclareMathOperator*{\argmaxdown}{arg\,max}
\DeclareMathOperator*{\argmindown}{arg\,min}
\DeclareMathOperator{\argmax}{arg\,max}
\DeclareMathOperator{\argmin}{arg\,min}

% cross-reference to other files
% the externaldocument should be compiled 
%   (and at least twice if you're using xr-hyper)
% \usepackage{xr-hyper}
% \usepackage{xr}

% --- external document (ordinary setting) ---
% \externaldocument{external_tex_file}
% --- end of ordinary setting ---

% --- external document (overleaf setting) ---
% externaldocument settings for Overleaf
% \makeatletter
% \newcommand*{\addFileDependency}[1]{% argument=file name and extension
%   \typeout{(#1)}
%   \@addtofilelist{#1}
%   \IfFileExists{#1}{}{\typeout{No file #1.}}
% }
% \makeatother
% \newcommand*{\myexternaldocument}[1]{%
%     \externaldocument{#1}%
%     \addFileDependency{#1.tex}%
%     \addFileDependency{#1.aux}%
% }
% \myexternaldocument{external_tex_file}
% --- end of overleaf setting ---

% mathtools can be used to define labeling format for equations
% one can use \eqref for a reference to a labeled equation.
\usepackage{mathtools}
\newtagform{supp}{(S-}{)}    % define a equation labeling format for suppliment
\usetagform{supp}            % use the supp format
\usetagform{default}         % use the default format

\usepackage{algorithmic}
\usepackage{algorithm}
\renewcommand{\algorithmicrequire}{\textbf{Input:}}
\renewcommand{\algorithmicensure}{\textbf{Output:}}

% package for hyperlinks
% It's error-prone because hyper link is quite difficult
% due to the fact the typesetting environment is complex.
% So you can disable this package and finish the document.
% Then sort out the hyperlink thing.
\usepackage[colorlinks,linkcolor=red,anchorcolor=blue,citecolor=green,CJKbookmarks=True]{hyperref}

% package for displaying highlighted codes
\usepackage{minted}

% package for input codec and output rendering font
\usepackage[utf8]{inputenc}    % it is always good practice to use utf8
                               % you can also try latin1, latin2, cp1252 and cp1250
\usepackage[T1]{fontenc}    % the default is T0, which only contains 128 characters
                            % you can try T1, T2A, T2B
% you can refer to https://www.overleaf.com/learn/latex/international_language_support#Font_encoding
% for more details.


\title{Multivariate Normal Distirbution}
\author{Chao Cheng}
\date{\today}



\begin{document}
\maketitle

For a multivariate normal distribution denoted by
\[
  \bm{x} \sim
  N\left(
    \bm{\mu}, \bm{\Sigma}
  \right)
  ,
\]
where $\bm{x} = \left(x_1, \cdots, x_n\right)^T\in\mathcal{R}^n$, $\bm{\mu} = \left(\mu_1, \cdots, \mu_n\right)^T$ is the mean vector and
\[
  \bm{\Sigma} = \left(
    \rho_{ij}\sigma_i\sigma_j
  \right)_{n\times n}
  =
  \begin{pmatrix}
    \sigma_1^2 & \cdots & \rho_{1n}\sigma_1\sigma_n    \\
    \vdots & \rho_{ij}\sigma_i\sigma_j & \vdots    \\
    \rho_{n1}\sigma_n\sigma_1 & \cdots & \sigma_n^2
  \end{pmatrix}
\]
is the $n\times n$ covariance matrix. Here $\sigma_1^2, \cdots, \sigma_n^2$ is the variance of $x_1, \cdots, x_n$ and $\rho_{ij}$ is the correlation between $x_i$ and $x_j$. Since it's symmetric, $\rho_{ij} = \rho_{ji}$ and $\rho_{ii} = 1$ for $i = 1, \cdots, n$.
\par
The density function is
\[
  f\left(\bm{x}\right) =
  \left(2\pi\right)^{-n / 2}
  \left|\bm{\Sigma}\right|^{-1 / 2}
  \mathrm{exp}\left(
    -\frac{1}{2}
    \left(\bm{x} - \bm{\mu}\right)^T
    \bm{\Sigma}^{-1}
    \left(\bm{x} - \bm{\mu}\right)
  \right)
  .
\]

\section{Bivariate normal distribution}
\label{sec:bivar-norm-distr}

For a bivarite normal variable $\left(x, y\right)$ following
\[
  \begin{pmatrix}
    x    \\
    y
  \end{pmatrix}
  \sim N\left(
    \begin{pmatrix}
      \mu_1    \\
      \mu_2
    \end{pmatrix},
    \begin{pmatrix}
      \sigma_1^2 & \rho\sigma_1\sigma_2    \\
      \rho\sigma_1\sigma_2 & \sigma_2^2
    \end{pmatrix}
  \right)
  ,
\]







\clearpage
\appendix

\section{R codes}
\begin{verbatim}
R codes
\end{verbatim}



\end{document}


%%%%%%%%%%%%%%%%%%%%%Subfigure
\begin{figure}[htb]
\centering
\begin{subfigure}[h]{0.49\textwidth}
\centering
\includegraphics[width=\textwidth]{5_23contour}
\caption{Contour Graphic}
\end{subfigure}
\begin{subfigure}[h]{0.49\textwidth}
\centering
\includegraphics[width=\textwidth]{5_23lattice}
\caption{Lattice Graphic}
\end{subfigure}
\caption{\label{fig:5.23.contour}}
\end{figure}

%%%%%%%%%%%%%%%%%%%%%TukeyHSD
\par
\makeatletter\def\@captype{figure}\makeatother
\begin{minipage}{.45\textwidth}
  \begin{figure}
    \centering
    \includegraphics[width=\textwidth]{5_10tukeyplot}
    \caption{Comparing Yield according to Pressure}
    \label{fig:5.10.tukeyplot}
  \end{figure}
\end{minipage}
\makeatletter\def\@captype{table}\makeatother
\begin{minipage}{.45\textwidth}
  \begin{table}
    \centering
    \begin{tabular}{rrrrr}
      \hline
      & diff & lwr & upr & p adj \\
      \hline
      215-200 & 0.32 & 0.11 & 0.52 & 0.00 \\
      230-200 & -0.18 & -0.39 & 0.02 & 0.08 \\
      230-215 & -0.50 & -0.70 & -0.30 & 0.00 \\
      \hline
    \end{tabular}
    \caption{Tukey multiple comparisons}
    \label{tab:5.10.tukeytable}
  \end{table}
\end{minipage}



%%%%%%%%%%%%%%%%%%%%%%%%custimize \item
\newcounter{Lcount}
\setcounter{Lcount}{0}
\begin{list}{2.1.\arabic{Lcount}}{\usecounter{Lcount}}
\item
\arabic 1, 2, 3 ...
\alph a, b, c ...
\Alph A, B, C ...
\roman i, ii, iii ...
\Roman I, II, III ...
\fnsymbol 星号,单剑号,双剑号等


\end{list}


%%% Local Variables:
%%% mode: latex
%%% TeX-master: t
%%% End:
