\documentclass[a4paper,12pt]{article}
\usepackage{geometry}
\geometry{left=2.5cm,right=2.5cm,top=2.5cm,bottom=2.5cm}
\renewcommand{\textfraction}{0.15}
\renewcommand{\topfraction}{0.85}
\renewcommand{\bottomfraction}{0.65}
\renewcommand{\floatpagefraction}{0.60}
\usepackage{amsmath}
\usepackage{amsfonts}
\usepackage{mathrsfs}
\usepackage{amsthm}
\usepackage{extarrows}
\usepackage{bm}
% \newcommand{\bm}{\symbfit}    % `bm` confilicts with `unicode-math`. In that case use \symbfit for bold math symbols
\usepackage{graphicx}
\usepackage[section]{placeins}
\usepackage{flafter}
\usepackage{array}
\usepackage{caption}
\usepackage{subcaption}
\usepackage{color}
\usepackage{multirow}
\usepackage{natbib}
% \usepackage{enumerate}
\usepackage{enumitem}    % more flexible than `enumerate` package, the reference will carry the whole label appearance, not just the counter, unlike the `enumerate` package.
\usepackage{upgreek}    % 'upgreek' letters

% \pdfstringdefDisableCommands{\let\bm=\relax}

\newtheorem{thm}{Theorem}
\newtheorem{cor}{Corollary}
\newtheorem{assum}{Assumption}
\newtheorem{rem}{Remark}
\newtheorem{lem}{Lemma}
\newtheorem{prop}{Proposition}

\setcounter{topnumber}{5}    % Maximum number of floats that can appear at the top of a text page; default 2. 
\setcounter{bottomnumber}{5}   % Maximum number of floats that can appear at the bottom of a text page; default 1. 
\setcounter{totalnumber}{10}    % Maximum number of floats that can appear on a text page; default 3. 

\DeclareMathOperator*{\argmaxdown}{arg\,max}
\DeclareMathOperator*{\argmindown}{arg\,min}
\DeclareMathOperator{\argmax}{arg\,max}
\DeclareMathOperator{\argmin}{arg\,min}

% cross-reference to other files
% the externaldocument should be compiled 
% (and at least twice if you're using xr-hyper)
% \usepackage{xr-hyper}
% \usepackage{xr}

% --- external document (ordinary setting) ---
% \externaldocument{external_tex_file}
% --- end of ordinary setting ---

% --- external document (overleaf setting) ---
% externaldocument settings for Overleaf
% \makeatletter
% \newcommand*{\addFileDependency}[1]{% argument=file name and extension
% \typeout{(#1)}
% \@addtofilelist{#1}
% \IfFileExists{#1}{}{\typeout{No file #1.}}
% }
%   \makeatother
%   \newcommand*{\myexternaldocument}[1]{%
%   \externaldocument{#1}%
%   \addFileDependency{#1.tex}%
%   \addFileDependency{#1.aux}%
% }
%   \myexternaldocument{external_tex_file}
%   --- end of overleaf setting ---

%   mathtools can be used to define labeling format for equations
%   one can use \eqref for a reference to a labeled equation.
\usepackage{mathtools}
\newtagform{supp}{(S-}{)}    % define a equation labeling format for suppliment
\usetagform{supp}            % use the supp format
\usetagform{default}         % use the default format

\usepackage{algorithmic}
\usepackage{algorithm}
\renewcommand{\algorithmicrequire}{\textbf{Input:}}
\renewcommand{\algorithmicensure}{\textbf{Output:}}

% package for hyperlinks
% It's error-prone because hyper link is quite difficult
% due to the fact the typesetting environment is complex.
% So you can disable this package and finish the document.
% Then sort out the hyperlink thing.
\usepackage[colorlinks,linkcolor=red,anchorcolor=blue,citecolor=green,CJKbookmarks=True]{hyperref}

% package for displaying highlighted codes
\usepackage{minted}

% package for input codec and output rendering font
\usepackage[utf8]{inputenc}    % it is always good practice to use utf8
% you can also try latin1, latin2, cp1252 and cp1250
\usepackage[T1]{fontenc}    % the default is T0, which only contains 128 characters
% you can try T1, T2A, T2B
% you can refer to https://www.overleaf.com/learn/latex/international_language_support#Font_encoding
% for more details.


\title{Minimum Detectable Difference (MDD) in Hypothesis Testing}
\author{Chao Cheng}
\date{\today}



\begin{document}
\maketitle
\tableofcontents{}

\section{Introduction}
\label{sec:introduction}

In this notes, we talk about minimum detectable difference (MDD) and other related concepts in hypothesis testing. First, we will use a test for normal mean when variance is known as an example. So for an i.i.d. sample $x_1, \cdots, x_n$ with $x_i \sim N\left(\mu, \sigma^2\right)$ with known $\sigma$, we want to test
\[
  H_0:\;\mu = \mu_0
  \quad
  \text{v.s.}
  \quad
  H_1:\;\mu\neq\mu_0
  .
\]
A two-sided test with significant level $\alpha$ can be constructed as to reject $H_0$ when
\[
  \left|
    \frac{\bar{x} - \mu_0}{\sqrt{\sigma^2 / n}}
  \right|
  \geq
  z_{1 - \alpha / 2}
  ,
\]
since $\bar{x} \sim N\left(\mu_0, \sigma^2 / n\right)$ under $H_0$. And for any given underlying $\mu$ that is not equal to $\mu_0$, the probability to correctly reject $H_0$, i.e., the power is computed as
\[
  \begin{aligned}
    & P\left(
      \left|
        \frac{\bar{x} - \mu_0}{\sqrt{\sigma^2 / n}}
      \right|
      \geq
      z_{1 - \alpha / 2}
    \right)    \\
    =&
    P\left(
      \bar{x} \geq \mu_0 + z_{1 - \alpha / 2} * \sqrt{\sigma^2 / n}
    \right)
    + P\left(
      \bar{x} \leq \mu_0 - z_{1 - \alpha / 2} * \sqrt{\sigma^2 / n}
    \right)    \\
    =& P\left(
      \frac{\bar{x} - \mu}{\sqrt{\sigma^2 / n}}
      \geq \frac{\mu_0 - \mu}{\sqrt{\sigma^2 / n}}
      + z_{1 - \alpha / 2}
    \right)
    +  P\left(
      \frac{\bar{x} - \mu}{\sqrt{\sigma^2 / n}}
      \leq \frac{\mu_0 - \mu}{\sqrt{\sigma^2 / n}}
      - z_{1 - \alpha / 2}
    \right)    \\
    =& \left(
      1 - \Phi\left(
        \frac{\mu_0 - \mu}{\sqrt{\sigma^2 / n}}
        + z_{1 - \alpha / 2}
      \right)
    \right)
    + \Phi\left(
      \frac{\mu_0 - \mu}{\sqrt{\sigma^2 / n}}
      - z_{1 - \alpha / 2}
    \right)
    .
  \end{aligned}
\]
And a confidence interval for $\mu$ is constructed as
\[
  \left[
    \bar{x} - z_{1 - \alpha / 2} * \sqrt{\sigma^2 / n}
    ,\quad
    \bar{x} + z_{1 - \alpha / 2} * \sqrt{\sigma^2 / n}
  \right]
  .
\]

So here we can see that in order to reject $H_0$, the critical point is the normalized \underline{difference} $\left|\frac{\bar{x} - \mu_0}{\sqrt{\sigma^2 / n}}\right|$ to be greater than $z_{1 - \alpha / 2}$. This critical point ($z_{1 - \alpha / 2}$) is determined by the type of test, which leads to the type of test statistics and the significant level. So in other words, as long as the observed difference $\left|\bar{x} - \mu_0\right|$ is greater than $z_{1 - \alpha / 2} * \sqrt{\sigma^2 / n}$, $H_0$ would be rejected no matter what the underlying $\mu$ is. So here $z_{1 - \alpha / 2} * \sqrt{\sigma^2 / n}$ is the minimum effect size (considering the sample size and variance) that would be just significant, hence the minimum detectable difference (MDD).

\section{MDD and other concepts}
\label{sec:mdd-other-concepts}









For more details one can see \citet{Mair2020p2109-2123}.


\bibliographystyle{plainnat}
\bibliography{../ref}





\end{document}


%%% Local Variables:
%%% mode: latex
%%% TeX-master: t
%%% End:
