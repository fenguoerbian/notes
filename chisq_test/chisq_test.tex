\documentclass[a4paper,12pt]{article}
\usepackage{geometry}
\geometry{left=2.5cm,right=2.5cm,top=2.5cm,bottom=2.5cm}
\renewcommand{\textfraction}{0.15}
\renewcommand{\topfraction}{0.85}
\renewcommand{\bottomfraction}{0.65}
\renewcommand{\floatpagefraction}{0.60}
\usepackage{amsmath}
\usepackage{amsfonts}
\usepackage{mathrsfs}
\usepackage{amsthm}
\usepackage{extarrows}
\usepackage{bm}
% \newcommand{\bm}{\symbfit}    % `bm` confilicts with `unicode-math`. In that case use \symbfit for bold math symbols
\usepackage{graphicx}
\usepackage[section]{placeins}
\usepackage{flafter}
\usepackage{array}
\usepackage{caption}
\usepackage{subcaption}
\usepackage{color}
\usepackage{multirow}
\usepackage{natbib}
% \usepackage{enumerate}
\usepackage{enumitem}    % more flexible than `enumerate` package, the reference will carry the whole label appearance, not just the counter, unlike the `enumerate` package.
\usepackage{upgreek}    % 'upgreek' letters

% \pdfstringdefDisableCommands{\let\bm=\relax}

\newtheorem{thm}{Theorem}
\newtheorem{cor}{Corollary}
\newtheorem{assum}{Assumption}
\newtheorem{rem}{Remark}
\newtheorem{lem}{Lemma}

\setcounter{topnumber}{5}    % Maximum number of floats that can appear at the top of a text page; default 2. 
\setcounter{bottomnumber}{5}   % Maximum number of floats that can appear at the bottom of a text page; default 1. 
\setcounter{totalnumber}{10}    % Maximum number of floats that can appear on a text page; default 3. 

\DeclareMathOperator*{\argmaxdown}{arg\,max}
\DeclareMathOperator*{\argmindown}{arg\,min}
\DeclareMathOperator{\argmax}{arg\,max}
\DeclareMathOperator{\argmin}{arg\,min}

% cross-reference to other files
% the externaldocument should be compiled 
%   (and at least twice if you're using xr-hyper)
% \usepackage{xr-hyper}
% \usepackage{xr}

% --- external document (ordinary setting) ---
% \externaldocument{external_tex_file}
% --- end of ordinary setting ---

% --- external document (overleaf setting) ---
% externaldocument settings for Overleaf
% \makeatletter
% \newcommand*{\addFileDependency}[1]{% argument=file name and extension
%   \typeout{(#1)}
%   \@addtofilelist{#1}
%   \IfFileExists{#1}{}{\typeout{No file #1.}}
% }
% \makeatother
% \newcommand*{\myexternaldocument}[1]{%
%     \externaldocument{#1}%
%     \addFileDependency{#1.tex}%
%     \addFileDependency{#1.aux}%
% }
% \myexternaldocument{external_tex_file}
% --- end of overleaf setting ---

% mathtools can be used to define labeling format for equations
% one can use \eqref for a reference to a labeled equation.
\usepackage{mathtools}
\newtagform{supp}{(S-}{)}    % define a equation labeling format for suppliment
\usetagform{supp}            % use the supp format
\usetagform{default}         % use the default format

\usepackage{algorithmic}
\usepackage{algorithm}
\renewcommand{\algorithmicrequire}{\textbf{Input:}}
\renewcommand{\algorithmicensure}{\textbf{Output:}}

% package for hyperlinks
% It's error-prone because hyper link is quite difficult
% due to the fact the typesetting environment is complex.
% So you can disable this package and finish the document.
% Then sort out the hyperlink thing.
\usepackage[colorlinks,linkcolor=red,anchorcolor=blue,citecolor=green,CJKbookmarks=True]{hyperref}

% package for displaying highlighted codes
\usepackage{minted}

% package for input codec and output rendering font
\usepackage[utf8]{inputenc}    % it is always good practice to use utf8
                               % you can also try latin1, latin2, cp1252 and cp1250
\usepackage[T1]{fontenc}    % the default is T0, which only contains 128 characters
                            % you can try T1, T2A, T2B
% you can refer to https://www.overleaf.com/learn/latex/international_language_support#Font_encoding
% for more details.


\title{Pearson's Chi-square Test}
\author{Chao Cheng}
\date{\today}



\begin{document}
\maketitle


There are mainly two types of situations that's suitable for a Pearson's Chi-square test. The first is to test one sample against a given vector, the so-called goodness-of-fit test. And the second is to test the existence of correlation between two samples, the so-called contingency/independence/association test.

\section{Effect size index $w$}
\label{sec:effect-size-index}

The Effect size index $w$ from Chapter 7 in \citet{Cohen2013p-} is
\begin{equation}
  \label{eq:effect_size_index}
  w = \sqrt{\sum\limits_{i = 1}^m
    \frac{
      \left(P_{1i} - P_{0i}\right)^2
    }{P_{0i}}}
  ,
\end{equation}
where
\begin{itemize}
\item $m$ is the number of cell.
\item $P_{0i}$ is the \textbf{propotion} in cell $i$ proposed by the null hypothesis.
\item $P_{1i}$ is the \textbf{propotion} in cell $i$ proposed by the alternative hypothesis and refects the effect for that cell.
\end{itemize}

\section{Test statistics $\chi^2$}
\label{sec:test-statistics-chi2}

The test statistic is just
\begin{equation}
  \label{eq:test_statistics}
  \chi^2_T = nw^2 = \sum\limits_{i = 1}^m
  \frac{\left(nP_{1i} - nP_{0i}\right)^2}{nP_{0i}}
  .
\end{equation}

\section{Goodness of fit test}
\label{sec:goodness-fit-test}

Let $\bm{x}\in\mathcal{R}^m$ be a sample from $multinomial(n, \bm{p})$ where $n$ is the number of trials and $\bm{p} = \left(p_1, \cdots, p_m\right)^T$ and $\sum\limits_{i = 1}^mp_i = 1$. Here we assume $p_i>0$ for all $i$ to eliminate some edge cases where some nomial is utterly impossible to happen. Then the probability of any given $\bm{x} = \left(x_1, \cdots, x_m\right)^T$ is
\[
  P\left(\bm{x} = \left(x_1, \cdots, x_m\right)^T\right) = \prod\limits_{i = 1}^mp_i^{x_i}
  ,
\]
where $\sum\limits_{i = 1}^mx_i = n$. This $\bm{x}$ can also be seen as the summation of $n$ samples $\bm{x}_1, \cdots, \bm{x}_n$ where each $\bm{x}_i\in\mathcal{R}^m$ follows $multinomial(1, \bm{p})$. And one and only one entry in each $\bm{x}_i$ is a single one while others $m-1$ entries all remain zero.
\par
Based on this observed $\bm{x}$, we want to test its underlying distribution $\bm{p}$ against a given vector $\bm{p}_0 = \left(p_{01}, \cdots, p_{0m}\right)^T$. And from \eqref{eq:test_statistics} we know that the test statistic is
\begin{equation}
  \label{eq:case1_test_statistic}
  \chi^2_T = \sum\limits_{i = 1}^m
  \frac{\left(x_i - np_{0i}\right)^2}{np_{0i}}
  .
\end{equation}

\subsection{Reject rule}
\label{sec:reject-rule}

Under null hypothesis, this test statistics follows a $\chi^2$ distribution with degree of freedom being $m - 1$. Proof for this statement can be found in Chapter 9 Pearson's chi-square test in David R. Hunter's \textbf{Notes for a graduate-level course in asymptotics for statisticians} \citep{Hunter2014p-}. And we reject $H_0$ when this test statistic $\chi^2_T$ is large enough.

\subsection{Power analysis}
\label{sec:power-analysis}

Under alternative hypothesis, i.e. $\bm{p}=\left(p_1, \cdots, p_m\right)^T\neq\bm{p}_0$. Denote $\bm{\delta} = \sqrt{n}\left(\bm{p} - \bm{p}_0\right)$ and $\bm{\Gamma} = \mathrm{diag}\left(\bm{p}_0\right)$. Then the test statistic now follows a \textbf{non-central chi-square distribution} with non-central parameter
\[
  \lambda = \bm{\delta}^T\bm{\Gamma}^{-1}\bm{\delta}
  .
\]
\par
\textbf{Non-central chi-square distribution}: Let $x_1, \cdots, x_n$ be independent normal distribution with means $\mu_1, \cdots, \mu_n$ and unit variance. Then $\sum\limits_{i = 1}^nx_i^2$ follows a non-central chi-square distribution with non-central parameter being
\[
  \lambda = \sum\limits_{i = 1}^n\mu_i^2
\]
and degree of freedom being $n$. And the pdf of $X = \sum x_i$ is given by
\[
  f\left(x;n,\lambda\right)
  = \mathrm{exp}\left(-\lambda / 2\right)
  \sum\limits_{i = 0}^\infty\frac{\left(\lambda / 2\right)^i}{i!}
  f_{n + 2i}\left(x\right)
  ,
\]
where $f_{n}\left(x\right)$ stands for the pdf of a ordinary chi-square distribution with $n$ degree of freedom. This result can also be found in Hunter's \textbf{Notes for a graduate-level course in asymptotics for statisticians} \citep{Hunter2014p-}. Also \citet{Guenther1977p83-83} and \citet{Meng1966p965-975} offers the same results.

\section{Contingency test}
\label{sec:contingency-test}

The same idea as that in Section~\ref{sec:goodness-fit-test} for the goodness of fit test except for that $\bm{p}_0$ is not now given, but rather computed based on \textbf{marginal proportion} of the data. So consider a $r\times c$ contingency table in Table~\ref{tab:contingency_table_count} and Table~\ref{tab:contingency_table_proportion}.
\begin{table}[htbp]
  \centering
  \begin{tabular}{|c|c|c|c|c|}
    \hline
    & col$_1$ & $\cdots $ & col$_c$ & Total \\
    \hline
    row$_1$ & $x_{11}$ & $\cdots$ & $x_{1c}$ & $x_{1\cdot} = \sum\limits_{j = 1}^cx_{1j}$    \\
    \hline
    $\vdots$ & $\vdots$ & $\ddots$ & $\vdots$ & $\vdots$    \\
    \hline
    row$_r$ & $x_{r1}$ & $\cdots$ & $x_{rc}$ & $x_{r\cdot} = \sum\limits_{j = 1}^cx_{rj}$    \\
    \hline
    Total & $x_{\cdot 1} = \sum\limits_{i = 1}^rx_{i1}$
              & $\cdots$
                          & $x_{\cdot c} = \sum\limits_{i = 1}^rx_{ic}$
                                    & $n = \sum\limits_{i = 1}^r\sum\limits_{j = 1}^cx_{ij}$    \\
    \hline
  \end{tabular}
  \caption{A contingency table, counts in cell}
  \label{tab:contingency_table_count}
\end{table}

\begin{table}[htbp]
  \centering
  \begin{tabular}{|c|c|c|c|c|}
    \hline
             & col$_1$ & $\cdots $ & col$_c$ & Total \\
    \hline
    row$_1$  & $p_{11} = x_{11}/n$ & $\cdots$ & $p_{1c} = x_{1c} / n$ & $p_{1\cdot} = x_{1\cdot} / n$    \\
    \hline
    $\vdots$ & $\vdots$ & $\ddots$ & $\vdots$ & $\vdots$    \\
    \hline
    row$_r$  & $p_{r1} = x_{r1} / n$ & $\cdots$ & $p_{rc} = x_{rc} / n$ & $p_{r\cdot} = x_{r\cdot} / n$    \\
    \hline
    Total    & $p_{\cdot 1} = x_{\cdot 1} / n$ & $\cdots$ & $p_{\cdot c} = x_{\cdot c} / n$ & $1$    \\
    \hline
  \end{tabular}
  \caption{A contingency table, proportion in cell}
  \label{tab:contingency_table_proportion}
\end{table}

The null hypothesis is that these two types of categories (arranged in row and column, respectively) is independent. Therefore the underlying distribution satisfies
\begin{equation}
  \label{eq:contingency_h0}
  p_{ij} = p_{i\cdot}p_{\cdot j},\quad 1\leq i \leq r,\quad 1\leq j \leq c.
\end{equation}
Then the alternative hypothesis is that there exists at least one $\left(i, j\right)$ such that \eqref{eq:contingency_h0} does not hold.

\subsection{Reject rule}
\label{sec:reject-rule-1}

Here the test statistic is
\[
  \chi^2_T =
  n\sum\limits_{i = 1}^r\sum\limits_{j = 1}^c
  \frac{\left(P_{1,ij} - P_{0,ij}\right)^2}{P_{0,ij}}
  = \sum\limits_{i = 1}^r\sum\limits_{j = 1}^c
  \frac{\left(
      x_{ij} - np_{i\cdot}p_{\cdot j}
    \right)^2}{np_{i\cdot}p_{\cdot j}}
  ,
\]
where $P_{1,ij}$ is just the observed proportion in cell $\left(i, j\right)$ and $P_{0,ij} = p_{i\cdot}p_{\cdot j}$ is the expected proportion computed based on marginal data.
\par
Under null hypothesis, $\chi^2_T$ follows a $\chi^2$ distribution with degree of freedom being $\left(r - 1\right)\left(c - 1\right)$. And $H_0$ is rejected for large value of $\chi^2_T$.

\subsection{Power analysis}
\label{sec:power-analysis-1}

The same as that in Section~\ref{sec:power-analysis}. Just now the $\bm{p}$ is length $r\times c$ instead of $m$, and the degree of freedom of the chi-square distribution is $\left(r - 1\right)\left(c - 1\right)$.

\section{Some conventional assumptions}
\label{sec:some-conv-assumpt}

\begin{itemize}
\item Simple random sample: i.i.d sample for each count/trial.
\item Sample size(whole table)
\item Expected cell count: no zero count. 5 or more in a cell of a 2-by-2 table, and 5 or more in 80\% of cells in larger table.
\end{itemize}

\section{Other related tests}
\label{sec:other-related-tests}

\begin{itemize}
\item For $2\times2$ table with small sample size, a Fisher's exact test can be considered.
\item For $2\times1$ table, a binomial test can be considered: Clopper-Pearson's test is an exact one, while the chi-square test or a normal test is a continuous approximation here.
\end{itemize}


\bibliographystyle{plainnat}
\bibliography{../ref}



\end{document}



%%% Local Variables:
%%% mode: latex
%%% TeX-master: t
%%% End:
