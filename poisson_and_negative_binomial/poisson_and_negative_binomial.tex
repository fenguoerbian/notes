\documentclass[a4paper,12pt]{article}
\usepackage{geometry}
\geometry{left=2.5cm,right=2.5cm,top=2.5cm,bottom=2.5cm}
\renewcommand{\textfraction}{0.15}
\renewcommand{\topfraction}{0.85}
\renewcommand{\bottomfraction}{0.65}
\renewcommand{\floatpagefraction}{0.60}
\usepackage{amsmath}
\usepackage{amsfonts}
\usepackage{mathrsfs}
\usepackage{amsthm}
\usepackage{extarrows}
\usepackage{bm}
% \newcommand{\bm}{\symbfit}    % `bm` confilicts with `unicode-math`. In that case use \symbfit for bold math symbols
\usepackage{graphicx}
\usepackage[section]{placeins}
\usepackage{flafter}
\usepackage{array}
\usepackage{caption}
\usepackage{subcaption}
\usepackage{color}
\usepackage{multirow}
\usepackage{natbib}
% \usepackage{enumerate}
\usepackage{enumitem}    % more flexible than `enumerate` package, the reference will carry the whole label appearance, not just the counter, unlike the `enumerate` package.
\usepackage{upgreek}    % 'upgreek' letters

% \pdfstringdefDisableCommands{\let\bm=\relax}

\newtheorem{thm}{Theorem}
\newtheorem{cor}{Corollary}
\newtheorem{assum}{Assumption}
\newtheorem{rem}{Remark}
\newtheorem{lem}{Lemma}

\setcounter{topnumber}{5}    % Maximum number of floats that can appear at the top of a text page; default 2. 
\setcounter{bottomnumber}{5}   % Maximum number of floats that can appear at the bottom of a text page; default 1. 
\setcounter{totalnumber}{10}    % Maximum number of floats that can appear on a text page; default 3. 

\DeclareMathOperator*{\argmaxdown}{arg\,max}
\DeclareMathOperator*{\argmindown}{arg\,min}
\DeclareMathOperator{\argmax}{arg\,max}
\DeclareMathOperator{\argmin}{arg\,min}

% cross-reference to other files
% the externaldocument should be compiled 
% (and at least twice if you're using xr-hyper)
% \usepackage{xr-hyper}
% \usepackage{xr}

% --- external document (ordinary setting) ---
% \externaldocument{external_tex_file}
% --- end of ordinary setting ---

% --- external document (overleaf setting) ---
% externaldocument settings for Overleaf
% \makeatletter
% \newcommand*{\addFileDependency}[1]{% argument=file name and extension
% \typeout{(#1)}
% \@addtofilelist{#1}
% \IfFileExists{#1}{}{\typeout{No file #1.}}
% }
%   \makeatother
%   \newcommand*{\myexternaldocument}[1]{%
%   \externaldocument{#1}%
%   \addFileDependency{#1.tex}%
%   \addFileDependency{#1.aux}%
% }
%   \myexternaldocument{external_tex_file}
%   --- end of overleaf setting ---

%   mathtools can be used to define labeling format for equations
%   one can use \eqref for a reference to a labeled equation.
\usepackage{mathtools}
\newtagform{supp}{(S-}{)}    % define a equation labeling format for suppliment
\usetagform{supp}            % use the supp format
\usetagform{default}         % use the default format

\usepackage{algorithmic}
\usepackage{algorithm}
\renewcommand{\algorithmicrequire}{\textbf{Input:}}
\renewcommand{\algorithmicensure}{\textbf{Output:}}

% package for hyperlinks
% It's error-prone because hyper link is quite difficult
% due to the fact the typesetting environment is complex.
% So you can disable this package and finish the document.
% Then sort out the hyperlink thing.
\usepackage[colorlinks,linkcolor=red,anchorcolor=blue,citecolor=green,CJKbookmarks=True]{hyperref}

% package for displaying highlighted codes
\usepackage{minted}

% package for input codec and output rendering font
\usepackage[utf8]{inputenc}    % it is always good practice to use utf8
% you can also try latin1, latin2, cp1252 and cp1250
\usepackage[T1]{fontenc}    % the default is T0, which only contains 128 characters
% you can try T1, T2A, T2B
% you can refer to https://www.overleaf.com/learn/latex/international_language_support#Font_encoding
% for more details.


\title{Poisson and Negative Binomial Distribution}
\author{Chao Cheng}
\date{\today}



\begin{document}
\maketitle

\tableofcontents{}

\section{Introduction}
\label{sec:introduction}

This notes is about poisson and negative binomial distribution, and their application in modeling count data.

\section{Poisson Distribution}
\label{sec:poisson-distribution}

\subsection{Basic Information}
\label{sec:basic-information}

The p.m.f of a random variable $X$ that follows Poisson distribution is
\[
  P\left(X = x\right) = \frac{\mathrm{e}^{-\lambda}\lambda^x}{x!}, \quad x = 0, 1, \cdots.
\]
The expectation and variance of $X$ is
\[
  \mathrm{E}X = \mathrm{Var}X = \lambda.
\]

\subsection{Some Additoinal Properties}
\label{sec:some-addit-prop}

\begin{enumerate}
\item Summation of independent poisson distributed variables is still poisson distributed. That is to say,
  \par
  if
  \[
    X_1\sim Poisson(\lambda_1),\quad
    X_2\sim Poisson(\lambda_2),\quad
    X_1 \perp X_2
    ,
  \]
  then
  \[
    X_1 + X_2 \sim Poisson(\lambda_1 + \lambda_2)
    .
  \]
\item Poisson distribution can be related to Gamma distribution, which is also mentioned in ``gamma\_and\_beta.pdf'' notes.
\item A poisson distributed variable can be seen as counts of customer to a store or occurence of some events. And the \textbf{time interval between each count} follows exponential distribution. 
\end{enumerate}

\section{Negative Binomial Distribution}
\label{sec:negat-binom-distr}

\subsection{Basic Information}
\label{sec:basic-information-1}

Denote $X$ the number of failing attempts just before reaching a total of $r$ times of success and $p$ the probability of success. If each attempt is independent of each other, then the distribution of $X$ is called a negative binomial distribution. And the p.m.f of $X$ is
\[
  P\left(X = x\right)
  = C_{x + r - 1}^xp^r\left(1 - p\right)^x
  = \frac{\left(x + r - 1\right)!}{\left(r - 1\right)!x!}p^r\left(1 - p\right)^x
  ,\quad
  x = 0, 1, \cdots .
\]
In this notation, $X$ is the number of failing attempts. In some cases, one will model the distribution of number of total attempts $X'$, which is just $X' = X + r$. 

\par

Previously, $r$ is the number of required success, which is naturally an positive integer. But one can actually generalize $r$ to positive real number using Gamma function. Then the p.m.f of $X$ is written as
\[
  P\left(X = x\right) = \frac{\Gamma(x + r)}{\Gamma(r)\Gamma(x + 1)}p^r\left(1 - p\right)^x
  ,\quad
  x = 0, 1, \cdots
  ,
\]
which is also called \textbf{Polya} distribution.

\par

The expectation of a negative binomial/Polya distribution is
\[
  \mathrm{E}X = \frac{r\left(1 - p\right)}{p}
  .
\]
The variance of a negative binomial/Polya distribution is
\[
  \mathrm{Var}X = \frac{r\left(1 - p\right)}{p ^ 2}
  .
\]

\subsection{Some Additional Properties}
\label{sec:some-addit-prop-1}

\begin{enumerate}
\item Summation of independent negative binomial distributed variables. That is to say, if $X_1 \sim NB\left(r_1, p\right)$, $X_2 \sim NB\left(r_2, p\right)$ and $X_1 \perp X_2$, then
  \[
    X_1 + X_2 \sim NB\left(r_1 + r_2, p\right)
    .
  \]
\item Relationship to Poisson and Gamma distribution.
  \begin{itemize}
  \item Related to poisson when $r\to\infty$.
  \item Related to poisson and gamma distribution.
  \end{itemize}
\item Dispersion parameter and other parametrization of negative binomial distribution.
  \begin{itemize}
  \item In some literature, expectation and variance are used to describe negative binomial distribution:
    \[
      \begin{aligned}
        \mathrm{E}X &= \mu    \\
        \mathrm{Var}X &= \sigma^2    \\
        r &= \frac{\mu^2}{\sigma^2 - \mu}    \\
        p &= \frac{\mu}{\sigma^2}
      \end{aligned}
    \]
  \item In some literature, $\alpha = \frac{1}{r}$ is called the dispersion parameter while others call $r$ the dispersion parameter. It also has other names such as ``shape parameter'', ``clustering coefficient'', ``heterogeneity'' and ``aggregation parameter''. Note that the expectation and the variance of a negative binomial distributed variable satisfies
    \[
      \sigma^2 = \mu + \alpha \mu^2 = \mu + \frac{\mu^2}{r}
      .
    \]
  \item Denote $\beta$ the \textbf{failure odds}, then
    \[
      \begin{aligned}
        \beta &= \frac{1- p}{p}    \\
        P\left(X = x\right) &=
        \frac{\Gamma\left(x + r\right)}{\Gamma\left(r\right)\Gamma\left(x + 1\right)}
        \left(\frac{\beta}{1 + \beta}\right)^r
        \left(\frac{1}{1 + \beta}\right)^x,
        \quad x = 0, 1, \cdots,    \\
        \mathrm{E}X &= r\beta    \\
        \mathrm{Var}X &= r\beta\left(1 + \beta\right).
      \end{aligned}
    \]
  \end{itemize}
\item Geometric distribution is a special case of negative binomial distribution.
\end{enumerate}

\section{Modeling Counts Data}
\label{sec:modeling-counts-data}

\subsection{Basic Modeling}
\label{sec:basic-modeling}

\subsection{Zero Inflation}
\label{sec:zero-inflation}



\bibliographystyle{plainnat}
\bibliography{../ref}


\end{document}



%%% Local Variables:
%%% mode: latex
%%% TeX-master: t
%%% End:
